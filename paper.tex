\documentclass{jsarticle}
%\setcounter{secnumdepth}{5}
\usepackage{amsthm}

%ユーザ自身による体裁のカスタマイズが許される状況なら以下が使える.
%奥村氏のjsarticleクラスやjsbookクラスを用いた場合にはこれがデフォルトになってる.
%\DeclareFontShape{JY1}{mc}{m}{n}{<->jis}{}
%\DeclareFontShape{JY1}{gt}{m}{n}{<->jisg}{}
%\DeclareFontShape{JT1}{mc}{m}{n}{<->jis-v}{}
%\DeclareFontShape{JT1}{gt}{m}{n}{<->jisg-v}{}

\begin{document}

\title{ゼミ形式の決定手続きに見る,複数回に渡る同一2択投票問題が持つ社会的選択理論的性質}
\author{一橋大学経済学部 2117272C 横山彪人}
\date{\today}
\maketitle

\begin{abstract}
  私の所属するゼミでは,その実施形式をオンラインにするか対面にするかを毎回メンバーによる
  投票で決めている.これは,複数人の選好を集約して1つの選択肢を選び取る社会的選択の問題である.
  そこで本論では,このゼミ形式の決定手続きが意見集約としていかなる性質を持っているかを述べる.
  その足掛かりとして,まずはボルダルールと戦略的操作に関する議論を紹介する.その後,ゼミの実施
  形式決定手続きの性質について議論する.そこでは,各個人は選択肢に対する強選好を持っており,
  各回の投票ではその時点で実現可能な選択肢の中で最も好ましいものを実現するような形式に投票する
  という仮定を置く.その元で,この手続きがペア全敗者を選ばないこと,いかなる2人以上の人数と
  2回以上のゼミの回数に対してもボルダ勝者と異なる選択肢を選び得ること,そして,回数に関する
  単峰性が成り立っているときにはその中位の回数に関連の深い選択肢が選ばれることを述べる.
\end{abstract}

%目次の出力
\tableofcontents
\clearpage

\section{はじめに}
\subsection{問題設定の背景}
\subsection{社会的選択理論とは何か}
\subsection{本誌の構成}

\section{多数決の問題点とベンチマークとしてのボルダルール}
\subsection{この節で述べること}

\subsection{多数決}

\subsection{ボルダルール}
\subsubsection{ボルダルールの優れた性質}
\subsubsection{ボルダルールの弱点}

\section{戦略的操作の可能性}
\subsection{この節で述べること}
\subsection{ギバート=サタスウェイト定理}
\subsection{中位投票者定理}

\section{ゼミ形式決定問題の性質}
\subsection{この節で述べること}
\subsection{ゼミ形式決定問題の形式化}
\subsection{ボルダルールとの比較}
\subsection{単峰性との関連性}

\section{まとめ}

\begin{thebibliography}{9}
  \bibitem{招待}
  坂井豊貴 (2018)『社会的選択理論への招待』日本評論社
  \bibitem{メカデザ}
  坂井豊貴他 (2020)『メカニズムデザイン』ミネルヴァ書房
  \bibitem{単峰性}
  Black, D. (1948) On the Rationale of Group Decision-making, Journal of Political Economy,
  Vol. 56, pp.23-34
  \bibitem{black}
  Black, D. (1976) Partial Justification of the Borda Count, Public Choice,
  Vol. 28, pp. 1-15
  \bibitem{Coughlin}
  Coughlin, P. (1979) A Direct Characterization of Black's First Borda Count,
  Economic Letters, Vol. 4, pp. 131-133
  \bibitem{borda}
  de Borda J-C (1784) M\'{e}moire sur les \'{e}lections au scrutin, Histoire de l'Acad\'{e}mie
  Royal des Science, pp. 657-664
  \bibitem{Farkas}
  Farkas, D. and Nitzan, S. (1979) The Borda Rule and Pareto Stability: A Comment,
  Econometrica, Vol. 47, pp. 1305-1306
  \bibitem{gibbard}
  Gibbard, A. (1973) Manipulation of voting schemes: A general result, Econometrica, Vol. 41,
  pp. 587-601
  \bibitem{revisit}
  Okamoto, N. and Sakai, T. (2019) The Borda rule and the pairwise-majority-loser
  revisited, Review of Economic Design, Springer; Scioety for Economic Design, vol 23(1)
  , pages 75-89, June
  \bibitem{satterthwaite}
  Satterthwaite, M.A. (1975) Strategy-proofness and Arrow's conditions:
  Existence and correspondence theorems for voting procedures and social
  welfare functions, Journal of Economic Theory, Vol. 10, pp. 187-217
\end{thebibliography}


\end{document}