\documentclass{jsarticle}
%\setcounter{secnumdepth}{5}
\usepackage{amsthm}

%ユーザ自身による体裁のカスタマイズが許される状況なら以下が使える.
%奥村氏のjsarticleクラスやjsbookクラスを用いた場合にはこれがデフォルトになってる.
%\DeclareFontShape{JY1}{mc}{m}{n}{<->jis}{}
%\DeclareFontShape{JY1}{gt}{m}{n}{<->jisg}{}
%\DeclareFontShape{JT1}{mc}{m}{n}{<->jis-v}{}
%\DeclareFontShape{JT1}{gt}{m}{n}{<->jisg-v}{}

\begin{document}

\title{ゼミ形式の決定手続きに見る,複数回に渡る同一2択投票問題が持つ社会的選択理論的性質}
\author{一橋大学経済学部 2117272C 横山彪人}
\date{\today}
\maketitle

\begin{abstract}
  ここにアブストラクトを書きます.
\end{abstract}

%目次の出力
\tableofcontents
\clearpage

\section{はじめに}
\subsection{問題設定の背景}

\subsection{本誌の構成}

\subsection{社会的選択理論とは何か}

\section{多数決の問題点とベンチマークとしてのボルダルール}
\subsection{この節で述べること}

\subsection{多数決}

\subsection{ボルダルール}
\subsubsection{ボルダルールの優れた性質}
\subsubsection{ボルダルールの弱点}

\section{戦略的操作の可能性}
\subsection{この節で述べること}
\subsection{ギバート=サタスウェイト定理}
\subsection{中位投票者定理}

\section{ゼミ形式決定問題の性質}
\subsection{この節で述べること}
\subsection{ゼミ形式決定問題の形式化}
\subsection{ボルダルールとの比較}
\subsection{単峰性との関連性}

\section{まとめ}

\begin{thebibliography}{9}
  \bibitem{Unbound}
  Alan Hoenig: \TeX{} UNBOUND \LaTeX{} \& \TeX{} strategies
  for fonts, graphics, \& more, Oxford University Press (1998).
\end{thebibliography}


\end{document}