\documentclass[dvipdfmx]{jsarticle}
%\setcounter{secnumdepth}{5}
\usepackage{otf}
\usepackage{array}
\usepackage{amsmath}
\usepackage{textcomp}
\usepackage{amssymb}
\usepackage{amsthm}

%ユーザ自身による体裁のカスタマイズが許される状況なら以下が使える.
%奥村氏のjsarticleクラスやjsbookクラスを用いた場合にはこれがデフォルトになってる.
%\DeclareFontShape{JY1}{mc}{m}{n}{<->jis}{}
%\DeclareFontShape{JY1}{gt}{m}{n}{<->jisg}{}
%\DeclareFontShape{JT1}{mc}{m}{n}{<->jis-v}{}
%\DeclareFontShape{JT1}{gt}{m}{n}{<->jisg-v}{}

\newtheorem{definition}{定義}[section]
\newtheorem{proposition}[definition]{命題}
\newtheorem{theorem}[definition]{定理}
\newtheorem{example}{例}[section]
\newtheorem*{THeorem}{定理}
\newtheorem*{PRoposition}{命題}
\newtheorem*{DEfinition}{定義}

\def\proofname{証明}

\renewcommand*{\theequation}{\thesection.\arabic{equation}}

% 倍角ダッシュの定義
\DeclareRobustCommand*\ddash{\makebox[2zw][s]{-\hss-\hss-}}

\begin{document}

\title{ゼミの形式決めの社会的選択関数としての定式化とその性質}
\author{}
\date{}
\maketitle

\begin{center}
  一橋大学経済学部 学士論文\\
  学籍番号:2117272C\\
  氏名:横山彪人\\
  ゼミナール指導教員:武岡則男
\end{center}

\vfill
\begin{abstract}
  著者の所属するゼミでは,その実施形式をオンラインにするか対面にするかを毎回メンバーによる
  投票によって決定している.これは複数人の選好を集約して1つの選択肢を選び取る社会的選択の問題である.
  
  本論文では,このゼミ形式の決定手続きが持つ意思集約方法としての性質を考察し,その結果を述べる.
  
  その足掛かりとして,はじめにボルダルールと戦略的操作に関する議論を紹介する.
  ボルダルールに関してはその利点と欠点について述べる.利点についてはペア全敗者を選ばないことを
  始めとした5つの利点を述べる.欠点については,結果に対する戦略的操作が可能なこと,
  さらに膨大な選択肢に対する選好を表明することが実用的でないことを述べる.
  
  次に,ボルダルールの欠点でも見られた戦略的操作に関する問題を,ボルダルール
  だけでなく意思集約方法一般に関する議論として紹介する.そこではまずギバード=サタスウェイト定理を
  引用し,ある条件を満たした非独裁的な意思集約方法は必ず戦略的操作が可能になってしまうことを述べる.次に
  選好の単峰性という概念を紹介し,すべての個人の選好がこの単峰性という条件を満たすときには戦略的操作を
  不可能にする意思集約方法が存在することを述べる.
  
  最後に,これらを踏まえた上でゼミの実施形式決定手続きが持つ意思集約方法としての性質を考察する.
  そのためにまずこの手続きを社会的選択関数として定式化する.この関数はまるで,各回の投票において各参加者が
  その時点で実現可能な選択肢のうち最も好ましいものに投票するかのようにふるまう.したがってこの関数は,
  最初に選択肢全体に関する選好を表明せずに各回ごとに投票を行うといった,実際に著者のゼミで行われている
  手続きの性質を反映している.さらにこの関数がいくつかの性質を満たすことを証明する.
  特に重要なのは,まずペア全敗者を選ばないこと.続いて,個々の選好が対面形式の実施回数に
  対して単峰性が成り立っているときは,その中位の回数に特徴付けられた選択肢が実現すること.
  そして,同じ条件下では,結果に対する戦略的な操作を部分的に制限できるということである.\\
\end{abstract}
\vfill

%目次の出力
\newpage
\tableofcontents
\clearpage

\section{問題設定の背景と社会的選択理論との関係}\label{sec:問題設定の背景と社会的選択理論との関係}
\subsection{問題設定の背景}
本論文が執筆された2020年は,COVID-19の感染拡大の影響で前期の講義がほぼすべてオンラインに移行した.しかし
後期になるとこの制限が緩和されたことで,ゼミに関しては対面形式での実施が可能になった.著者の所属するゼミでは
オンラインと対面を併用し,毎回次のゼミの実施形式のみをオンラインか対面かの2択投票の多数決で決めることになった.

この手続きは複数人の選好を集約して1つの選択肢を選び取る意思集約方法の1つである.そしてその方法の良し悪しを
分析するのは社会的選択理論という分野の問題である.本論文執筆の動機は,著者が実際に使用している意思集約方法の性質を,
社会的選択理論的な見地から明らかにしたいという思いによる.

\subsection{社会的選択理論とは何か}\label{社会的選択とは何か}
社会的選択理論というと「投票における意思集約方法の設計」と「社会状態の望ましさを評価する基準の構築」
を主に含むことが多い(坂井 2018, p. 3).そのうち本論文と関わりが深いのは前者の方である.つまり
2人以上の個人で1つの意思決定をする際に,どのように異なる選好同士の折り合いをつけ,意思を1つにまとめ上げるのが
よいのか,その意思集約方法の理論的性質に注目する.ここで,そのモチベーションを共有するために,意思集約方法の
性質が重要になる例を2つ述べる.

1つは多数決による例である.ある中学生のクラスで授業が休みになり,その時間を自由に使ってよい状況を考える.
ただし担任の教員には監督責任があるので,クラスの全員が同じ場所にいなければならない.このクラスには全員で30人
の生徒がいる.そのうち18人は体を動かして遊びたいと思っており,残る少数派の12人は教室で静かに読書をしたいと思っている.
このとき,「校庭でサッカーをして過ごす」という選択肢と「教室で読書をして過ごす」という選択肢で多数決を行えば,
前者が18人の指示を得て多数決に勝利する.一方でこの2つの選択肢に加えて「体育館でバスケットボールを
して過ごす」という選択肢が存在し,体を動かして遊びたい18人の生徒がサッカーとバスケットボールの選択肢に
半々に割れた場合には,多数決の勝者は「教室で読書をして過ごす」になる.この結果は全体における多数派である,
体を動かして遊びたい18人にとっては一番望ましくない選択肢であろう.これは,似たような選択肢の存在によって
票の割れが発生し,結果として少数派の意見が採用される事例である.この場合,多数決は意思集約方法として
好ましいだろうか,そうでないならば,他に優れた方法はあるだろうか.

もう1つはMalkevitch(1990)によって提示された次の例で,意思集約方法の選択が最終的な結果を変えてしまうものである.
まず設定として,55人の有権者が5つの選択肢$a,\ b,\ c,\ d,\ e$について,表~\ref{tab:Malkevitch}のような
選好を持っているとし,意思集約方法として次の4つを考える.

1つ目は多数決である.この場合は$a$が18票を得て勝利する.

\begin{table}[h]
  \caption{Malkevitch(1990)の例}\label{tab:Malkevitch}
  \begin{center}
    \begin{tabular}{c|c|c|c|c|c|c}
      & 18人 & 12人 & 10人 & 9人 & 4人 & 2人 \\ \hline
    1位 & a & b & c & d & e & e \\
    2位 & b & e & b & c & b & c \\
    3位 & e & d & e & e & d & d \\
    4位 & c & c & d & b & c & b \\
    5位 & b & a & a & a & a & a \\
\end{tabular}
  \end{center}
\end{table}

2つ目,通常の多数決をしたあと,その勝利者が過半数の指示を得ていない場合は2位と決選投票を行う方法を採用すればどうなるだろうか.
多数決で1位の$a$が獲得した18票は過半数ではないので,2位の$b$と決選投票が行われる.表~\ref{tab:Malkevitch}
によれば,$a$を$b$より好む者は18人おり,その逆は37人いるので,最終的には$b$が勝利する.これは多数決と
異なる選択肢を選び取っている.

3つ目は,毎回の多数決で最少票の選択肢を消去していく方法である.すると1段回目では6票しか集まらなかった
$e$が消去され,2段回目では9票しか集まらなかった$d$が消去され,3段回目では16票しか集まらなかった$b$
が消去され,最後の段階では18票しか集まらなかった$a$が消去されて$c$が残る.これは前の2つのいずれとも
異なる選択肢を選び取っている.

4つ目は,順位ごとに重みづけの得点を与えて,その総得点が一番高いものを選ぶ方法である.ここでは1位に5点,
2位に4点,\ldots,5位に1点を与えることにする.このとき,選択肢を引数にその総得点を返す関数を$p(\cdot)$
と表すと,それぞれの総得点は
\begin{align*}
  p(a) &= 5 \times 18 \ +\  4 \times 0 \ +\ 3 \times 0 \ +\  2 \times 0\ +\ 1 \times 37 = 127 \\
  p(b) &= 5 \times 12 \ +\  4 \times 14 \ +\ 3 \times 0 \ +\  2 \times 11 \ +\ 1 \times 18 = 156 \\
  p(c) &= 5 \times 10 \ +\  4 \times 11 \ +\ 3 \times 0 \ +\  2 \times 34 \ +\ 1 \times 0 = 162 \\
  p(d) &= 5 \times 9 \ +\  4 \times 18 \ +\ 3 \times 18 \ +\  2 \times 10 \ +\ 1 \times 0 = 191 \\
  p(e) &= 5 \times 6 \ +\  4 \times 12 \ +\ 3 \times 37 \ +\  2 \times 0 \ +\ 1 \times 0 = 189
\end{align*}
のように計算されるので,総得点が一番高いのは$d$となる.この結果は前の3つのいずれとも異なるものである.

この例が示すように,有権者の選好が変わらなくても,その集約方法を変えると最終的な結果も変わってきてしまう.
最初の中学生による多数決の例とこのMalkevitchの例が示唆するものは,どの集約方法を採用するべきか,
採用される集約方法はどのような性質を満たしているべきかという問題が重要だということである.

\subsection{本論文の構成}
2節では意思集約方法のベンチマークとして多数決とボルダルールを取り上げる.多数決に関しては,ペア全敗者を
選んでしまうという問題を抱えていることを述べる.ボルダルールに関しては,ペア全敗者を選ばないことをはじめ
とする利点といくつかの欠点について述べる.

3節では意思の集約結果に対する戦略的な操作の可能性について取り上げる.
まず有名な不可能性定理であるギバード=サタスウェイト定理を引用して,全射性という性質を持つ意思集約方法
に対して,戦略的操作を不可能にすることを求めるならばそれは独裁制しかないことを述べる.そのあと,
個々の選好が単峰性という条件を満たすときには,戦略的操作を不可能にするような非独裁的な集約ルールが
存在することを述べる.

4節では,2,3節の内容を踏まえながら,著者のゼミが採用している意思集約方法が持つ性質について
議論する.具体的にはこの方法を関数として定式化したあと,その関数がペア全敗者を選ばないこと,
各々の選好が回数に関する単峰性という性質を満たすときには,その回数の中位に特徴付けられた
選択肢が選ばれること,そして同じ条件下で,結果に対する戦略的な操作を部分的に制限できることなど
を示す.

\section{多数決の問題点とベンチマークとしてのボルダルール}\label{sec:多数決の問題点とベンチマークとしてのボルダルール}
\subsection{はじめに}
この\ref{sec:多数決の問題点とベンチマークとしてのボルダルール}~節では,
意思集約方法の例示として,また4節でゼミ形式決定問題を定式化した際のベンチマークとして
多数決とボルダルールの2つを取り上げ,その性質について述べる.
\ref{subsec:多数決の問題点}~節では,多数決がペア全敗者を選んでしまうという問題点を,
ペア全敗者の定義と共に例を用いて解説する.続いて\ref{subsec:ボルダルール}~節ではボルダルールに
関する議論を紹介する.その中でまず\ref{subsubsec:ボルダルールの定義}~節ではボルダルールの定義を述べる.
\ref{ボルダルールの利点}~節ではボルダルールが持つ優れた性質を5つ紹介する.\ref{ボルダルールの欠点}~節
ではボルダルールの欠点を2つ述べる.

\subsection{多数決の問題点}\label{subsec:多数決の問題点}
多数決は,恐らく最も有名な意思集約方法の一つである.その方法は,各有権者が1つの選択肢に投票し,
最多票を得た選択肢が勝者になるというものである.しかしこの多数決がある問題を抱えていることは
Borda(1784)が指摘していた.彼が指摘したのは次のようなことである.今,3つの選択肢$x,\ y,\ z$に対する
21人の選好が表~\ref{ペア全敗者}のような状況を考える.多数決を行えば$x$が最多票の8票を得て勝者
となる.しかしここで1対1,つまりペアごとの多数決を行えば$x$は$y$に8対13で負け,$z$にも8対13で
負ける.ペアごとの多数決で全敗する$x$をペア全敗者\footnote{この命名は坂井(2018)に基づく}と呼ぶ.
多数決にはこのように,ペア全敗者を選びうるという性質がある.

\begin{table}[h]
  \caption{ペア全敗者を選ぶ多数決}\label{ペア全敗者}
  \begin{center}
    \begin{tabular}{c|c|c|c|c} \hline
      & 4人 & 4人 & 7人 & 6人 \\ \hline
      1位 & $x$ & $x$ & $y$ & $z$ \\
      2位 & $y$ & $z$ & $z$ & $y$ \\
      3位 & $z$ & $y$ & $x$ & $x$ \\ \hline 
    \end{tabular}
  \end{center}
\end{table}



\subsection{ボルダルール}\label{subsec:ボルダルール}
\subsubsection{ボルダルールの定義}\label{subsubsec:ボルダルールの定義}
多数決がペア全敗者を選びうるという問題を受けて,Bordaはボルダルールと呼ばれる次の集計方法を提案した.
それは,選択肢が$m$個あるときに,それぞれの有権者が1位に$m$点,2位に$m-1$点,\ldots,$m$位に1点を
与え,その総得点の一番高いものを選び取る方法である.\<\footnote{これは\ref{社会的選択とは何か}で述べた
Malkevitchの例の4つ目の方法である.}\ 
この総得点をボルダ得点,ボルダ得点の一番高い選択肢をボルダ勝者と呼ぶことにする.

選択肢を引数としてそのボルダ得点を返す関数を$p(\cdot)$と表すと,先の表の~\ref{ペア全敗者}における
$x,\ y,\ z$のボルダ得点はそれぞれ,
\begin{align*}
  p(x) = 3 \times 8 \ + \ 2 \times 0 \ + \ 1 \times 13 = 37 \\
  p(y) = 3 \times 7 \ + \ 2 \times 10 \ + \ 1 \times 4 = 45 \\
  p(z) = 3 \times 6 \ + \ 2 \times 11 \ + \ 1 \times 4 = 44
\end{align*}
となり,$y$がボルダ勝者となる.多数決を採用すれば選ばれていた$x$は,
ボルダルールでは最小得点になっている.

\subsubsection{ボルダルールの利点}\label{ボルダルールの利点}
ボルダルールには様々な優れた性質がある.ここではそのうちの5つを簡単に紹介する.

まず1つ目に,ボルダルールはペア全敗者を選ばない.これは多数決がペア全敗者を選んでしまうことを受けて
提案された方法としては当然満たして欲しい性質である.証明は次の2つ目の性質の系として直ちに導かれる.

2つ目は,1対1の多数決でボルダ勝者が他のある選択肢に負ける場合,少なくとも1つの選択肢に1対1の多数決
で勝てるというものである(Okamoto and Sakai 2019).

3つ目はボルダルールに留まらずスコアリングルール全般に関わる定理である.ここでスコアリングルールとは,
より高い順位には高い得点を与えるように重みづけをして,その最終的な総得点の1番高いものを選び取るルールの
ことを指す.そしてこのスコアリングルールに関して,どんな有権者数でもペア全敗者を選ぶことがないスコアリングルール
はボルダルールのみであることが示されている(Okamoto and Sakai 2019).
つまりボルダルール以外のどんなスコアリングルールについても,ある人数とある選好組が存在して,
その元でペア全敗者を選んでしまうというわけである.
これは無数にあるスコアリングルールの中でボルダルールの唯一性を示す定理と言える.

4つ目は全員一致までの近さという観点から集約方法を分析し,全員一致まで最も近い選択肢がボルダルールに
よって選ばれるというものである(Farkas and Nitzan 1979).

そして5つ目は,1対1での多数決に注目し,その総当たり戦での平均得票率が最大になる選択肢をボルダルールが
選び取るというものである(Black 1976, Coughlin 1979).4つ目と同じでこちらも全員一致までの近さを測っているという見方ができる.
しかし4つ目は全体のランキング表における選択肢の位置を使用しているのに対して,この5つ目はあくまで
1対1の比較に基づいて全員一致までの近さを測っているところに違いがある.

\subsubsection{ボルダルールの欠点}\label{ボルダルールの欠点}
以上見たようにボルダルールには様々な優れた性質があるが,問題点もある.ここではそのうち2つを述べる.

まず,ボルダルールは参加者に対してすべての選択肢に対するランクづけを要請する.選択肢が3つや
4つのときには問題ないかもしれないが,選択肢の数がさらに増えたときにはこのランクづけは参加者にとって
ひどく億劫な作業になる.特に本論文で考えたいゼミ形式の決定に関する問題では,ゼミの回数が$m$回として
各回において対面かオンラインかの2択があるから,
選択肢の数は$2^m$個にものぼる.後期のゼミは少なくとも10回はあるので選択肢の数は1000を超えることになる.
このような膨大な量の選択肢に対する選好を表明させるのは現実的ではない.したがってボルダルールをそのまま
ゼミ形式の決定問題に適用することはできない.

次に,ボルダルールは戦略的操作の影響を受ける.これは坂井(2018)の例を用いて次のように
説明される.今,4つの選択肢$x,\ y,\ z,\ w$に対して,3人の個人が表~\ref{Sakai2018}のような
選好を持っているとする.このときの4つのボルダ得点はそれぞれ,$p(x) = 10,\ p(y) = 11,
\ p(z) = 6, \ p(w) = 3$となりボルダ勝者は$y$である.ところがここで,$y$よりも$x$を
高く順序づけている佐藤が1位から順に$xzwy$という虚偽の選好を表明すれば,4つのボルダ得点は
それぞれ,$p(x) = 11,\ p(y) = 9,\ p(z) = 7, \ p(w) = 4$となり,佐藤にとってより望ましい
$x$を実現することができる.

\begin{table}[h]
  \caption{坂井(2018)の例}\label{Sakai2018}
  \begin{center}
    \begin{tabular}{c|c|c|c} \hline
        & 佐藤 & 高橋 & 中野 \\ \hline
      $1$位 & $x$ & $y$ & $y$ \\
      $2$位 & $y$ & $x$ & $x$ \\
      $3$位 & $z$ & $z$ & $z$ \\
      $4$位 & $w$ & $w$ & $w$ \\ \hline
    \end{tabular}
  \end{center}
\end{table}

今見たように,ボルダルールには,虚偽の選好を表明することで自分にとってより好ましい結果を
選ばせることができるような状況が存在する.逆に,いかなる状況,いかなる個人においても,虚偽の選好を
表明することで結果をより好ましいものに変えることができないとき,その意思集約方法は
耐戦略性を満たすという.ボルダルールは耐戦略性を満たさないということだ.


\section{戦略的操作の可能性}\label{sec:戦略的操作の可能性}
\subsection{はじめに}
\ref{ボルダルールの欠点}~節ではボルダルールの結果に対する戦略的操作が可能な場合があること,つまり
ボルダルールが耐戦略性を満たさないという問題があることを述べた.この\ref{sec:戦略的操作の可能性}~節では,
耐戦略性に関する議論の対象をボルダルールだけでなく一般の意思集約方法にまで広げる.
\ref{subsec:ギバード=サタスウェイト定理}~節では,ギバード=サタスウェイト定理を引用して,
全射性という弱い条件を満たす意思集約方法に耐戦略性を求めるならばその方法は独裁制しかないことを述べる.
\ref{subsec:単峰性と中位選択関数}~節では,選好に関する単峰性という概念を紹介し,
すべての個人の選好が単峰性という条件を満たす場合には,耐戦略性を満たす非独裁的な
意思集約方法が存在することを述べる.この単峰性に関する発想は4節で応用される.

\subsection{ギバード=サタスウェイト定理}\label{subsec:ギバード=サタスウェイト定理}
\ref{ボルダルールの欠点}~節で述べたように,ボルダルールは耐戦略性を満たさない.
しかしこの問題点はボルダルールだけでなく,ほぼすべての民主的な意思集約方法に当てはまることが
Gibbard(1973)やSatterthwaite(1975)によって知られている.ここではギバード=サタスウェイト定理
として知られるその成果を定式化された形で述べる.そのためにいくつかの記法を導入し,この定理に
関連する定義を述べる.尚,以降定義などで用いる記法は坂井他(2020)に基づいた.

まず,今まで意思集約方法と呼んでいた対象は社会的選択関数という言葉を用いて次のように定義することができる.

\begin{definition}[社会的選択関数]
  $I$を個人の集合,$X$を選択肢の集合,$X$上の選好の集合を$\mathcal{R}$,
  $X$上の強選好の集合を$\mathcal{P}$とする.各$i \in I$の選好を$\succsim_i$で表す.
  個人$i$が取りうる選好の集合を$\mathcal{D}_i \subset \mathcal{R}$と表し,これを
  選好集合と呼ぶ.個人の選好集合の直積
  \[
    \mathcal{D}_I \stackrel{\mathrm{def}}{=} \mathcal{D}_1 \times
    \mathcal{D}_2 \times \cdots \times \mathcal{D}_n
  \]
  をドメインと呼び,個人の選好組を
  \[
    \succsim \stackrel{\mathrm{def}}{=} (\succsim_1, \succsim_2, \ldots, \succsim_n)
      \in \mathcal{D}_I
  \]
  で表す.このとき社会的選択関数とは以下で表される関数のことである.
  \[
    {\mathnormal{f}}\colon \mathcal{D}_{\mathrm{I}} \to \mathrm{X}
  \]
\end{definition}

社会的選択関数は,選択肢の集合に対する各個人のランキング表を受け取り,それらを何らかの方法で集計して
ただ1つの選択肢を選び取る.この一連のアルゴリズムはまさに社会の意思を集約するルールそのものである.
ここで定義された言葉を用いて社会的選択関数の様々な性質もまた定義することができる.まず,
先に述べた耐戦略性は次のように定式化される.続いて独裁制と全射性についての定義も与える.

\begin{definition}[耐戦略性]
  社会的選択関数$f$が耐戦略性を満たすとは,すべての$i \in I$,$\succsim\ \in \mathcal{D}_I$,
  $\succsim_{i}'\ \in \mathcal{D}_i$について
  \begin{equation}\label{eq:耐戦略性}
    f(\succsim) \succsim_{i} f(\succsim_{i}',\succsim_{-i})
  \end{equation}
  が成り立つことである.
\end{definition}

この定義における$\succsim_{-i}$という記号は,$I$に属するすべての個人の選好組$\succsim$から
$i \in I$の選好$\succsim_i$を抜いたもの,つまり$n-1$人の選好組$(\succsim_1, \ldots, \succsim_{i-1},\ 
\succsim_{i+1}, \ldots, \succsim_n)$である.したがって$f(\succsim)$と$f(\succsim_i', \succsim_{-i})$
を書き下すと
\begin{align*}
  f(\succsim) =& \hspace{5mm} f(\succsim_1, \ldots, \succsim_{i-1},\ \succsim_i,\ \succsim_{i+1}, \ldots, \succsim_n) \\
  f(\succsim_i', \succsim_{-i}) =& \hspace{5mm} f(\succsim_1, \ldots, \succsim_{i-1},\ \succsim_i',\ \succsim_{i+1}, \ldots, \succsim_n)
\end{align*}
になる.すなわち$f(\succsim_{i}', \succsim_{-i})$は,個人$i$だけが別の選好を表明したときにこの社会的選択関数$f$に
選ばれる選択肢を表している.それが元の選好組$\succsim$における個人$i$の選好$\succsim_i$で測って$f(\succsim)$
より好ましくないというわけだから,個人$i$は虚偽の選好を表明しても得ができないということが(\ref{eq:耐戦略性})によって
表現されている.

\begin{definition}[独裁制]\label{def:独裁制}
  社会的選択関数$f$が独裁制であるとは,ある個人$i \in I$が
  存在し,すべての$\succsim\  \in \mathcal{D}_I$について
  \begin{equation}\label{eq:独裁制}
    f(\succsim)\succsim_i x\hspace{8pt} \forall x \in \mathrm{X}  
  \end{equation}
  が成り立つことである.この$i$を独裁者という
\end{definition}

この定義の意味するところを述べる.独裁者$i \in I$に対して(\ref{eq:独裁制})が成り立つということは,
$x \in X$が独裁者$i$にとって最も好ましい選択肢$b \in X$であっても成り立っていなければならない.つまり
このとき(\ref{eq:独裁制})は$f(\succsim) \succsim_i b$である.この式を満たす$f(\succsim) \in X$は
独裁者$i$にとって最も好ましい選択肢$b$か$b$と無差別な$c \in X$である.したがってこの社会的選択関数は,
どんな選好組を受け取ろうが独裁者$i$にとっても最も好ましい選択肢を返す.これはまさしく独裁制である.


\begin{definition}[全射性]
  社会的選択関数が全射性を満たすとは,任意の$x \in \mathrm{X}$について,ある$\succsim \ \in \mathnormal{D}_{\mathrm{I}}$
  が存在して$\mathnormal{f}(\succsim) = x$
  が成り立つことである
\end{definition}

この定義の意味するところは関数の全射性と同じである.つまりどんな選択肢$x \in X$であっても個人の選好組$\succsim$
次第ではそれを実現することができるということである.

以上定義した耐戦略性,独裁制,全射性によって,ギバード=サタスウェイト定理を正式な形式で述べることができる.

\begin{theorem}[ギバード=サタスウェイト定理]\label{thm:ギバード=サタスウェイト定理}
  $|X| \geq 3$かつ$\forall i \in I \ \mathcal{D}_i = \mathcal{P}$とする.社会的選択関数
  $f\colon \mathcal{D}_{I} \to X$が全射性を満たすとき以下が成り立つ.
  \begin{center}
    $f$は耐戦略性を満たす $\Leftrightarrow$ $f$は独裁制である.
  \end{center}
\end{theorem}

\ref{subsec:ギバード=サタスウェイト定理}~節の冒頭で述べたように,この定理の示唆するところは次である.
それはつまり,強選好ドメイン上で定義された全射性を満たす社会的選択関数に耐戦略性を求めるならば,
その意思集約ルールは独裁制しかないということである.我々が自由に設定できるのはせいぜい誰を独裁者
にするのかくらいしかない.


\subsection{単峰性と中位選択関数}\label{subsec:単峰性と中位選択関数}
本論文における耐戦略性の議論は,ボルダルールがこれを満たさないということから始まった.しかし,
一般的に言って,社会的選択関数に耐戦略性を要請することは極めて難しいということが
ギバード=サタスウェイト定理によって示された.

ただし個人の選好が単峰性という条件を満たすときには,耐戦略性を満たす非独裁的な
社会的選択関数が存在することがBlack(1948)で指摘されている.ここでは単峰性とその関数に関する
議論を紹介し,次節でゼミの形式決定問題に援用する.

\begin{definition}[単峰性]
  選択肢の集合を$A = \{a_1, a_2, \ldots, a_{l}\}$と表し,各選択肢は区間$[0,1]$内の数値として
  与えられ,
  \[
    0 \leq a_1 < a_2 < \cdots < a_{l} \leq 1
  \]
  を満たすとする.$A$上の選好$\succsim_i \ \in \mathcal{R}$が単峰的であるとは,ベストの選択肢
  $b(\succsim_i) \in A$がただ一つ存在して,任意の$k, k' \in \{ 1, 2, \ldots, \mathnormal{l} \}$
  について
  \[
    a_k \neq b(\succsim_i) \Rightarrow b(\succsim_i) \succ_i a_k
  \]
  かつ
  \[
    [a_{k'} < a_{k} < b(\succsim_i) \text{または} b(\succsim_i) < a_{k} < a_{k'}]
    \Rightarrow b(\succsim_i) \succ_i a_{k} \succ_i a_{k'}
  \]
  を満たすことである.また,$A$上の単峰的な選好すべてからなる集合を$\mathcal{T}$と書く
\end{definition}

この単峰性の直観的な理解は次のようなものである.すなわち選択肢をなんらかの基準で直線上に並べることができ,
その中で最も好ましいものが存在する.各選択肢は,この最も好ましいものから離れれば離れるほど
望ましくなくなっていくというものだ.単峰性が成り立ちそうな選択肢の例としては,税率,売り手にとっての商品の値段,
部屋の室温などが考えられる.

すべての個人の選好が単峰的であるとき,中位選択関数を定めることができる.そしてこの中位選択関数が
耐戦略性を満たす非独裁的な社会的選択関数である.

\begin{definition}[中位選択関数]
  任意の$i$に対して$\mathcal{D}_i \subset \mathcal{T}$であるような
  選好組$\succsim \ \in \mathcal{D}_I$に対して,$b^m(\succsim)$を
  $b(\succsim_1)$,$b(\succsim_2)$,\ldots,$b(\succsim_n)$における中位の選択肢として定める.
  つまり,$b^m(\succsim) \in \{ b(\succsim_i):i \in I \}$かつ
  \begin{eqnarray*}
    | \{i \in I: b(\succsim_i) \leq b^m(\succsim) \} | \geq \frac{n}{2} \\
    | \{i \in I: b(\succsim_i) \geq b^m(\succsim) \} | \geq \frac{n}{2}
  \end{eqnarray*}
  が成り立つ.$n$が偶数の場合は中位が$2$つ存在するケースがあるが,その時は常に左側(または右側)の中位
  を選ぶことにする.このとき中位選択関数$f^m\colon \mathcal{D}_{I} \to X$
  を,$f^m(\succsim) \stackrel{\mathrm{def}}{=} b^m(\succsim)$と定める.
\end{definition}

\begin{theorem}
  任意の$i$に対して$\mathcal{D}_i \subset \mathcal{T}$であるとする.このとき中位選択関数
  $f^m\colon \mathcal{D}_I \to X$は耐戦略性を満たす.
\end{theorem}

\section{ゼミ形式決定問題の性質}\label{sec:ゼミ形式決定問題の性質}
\subsection{はじめに}
この\ref{sec:ゼミ形式決定問題の性質}節では,著者のゼミで実際に行っているゼミの実施形式の決定問題の性質に
ついて考察し,得られたいくつかの命題を証明する.\ref{subsec:ゼミ形式決定問題の定式化}~節では
ゼミ形式の決定問題を「ゼミ形式決定手続き」という名前の社会的選択関数として定式化する.
また,この関数が実際に行われている手続きとどのような関連があるかについても述べる.
\ref{subsec:ボルダルールとの比較}~節では,ゼミ形式決定手続きとボルダルールを比較し,
得られた2つの命題を証明する.1つ目はゼミ形式決定手続きがペア全敗者を選ばないこと,
2つ目は,人数が2人以上かつゼミの回数が2回以上なら必ず,ボルダルールと異なる選択肢を選び取るような
選好組$\succsim$が存在することである.\ref{subsec:単峰性との関連}~節では,
\ref{subsec:単峰性と中位選択関数}~節で取り上げた単峰性の概念を直和に分解された集合上に拡張する.
具体的には,個人の選好に対して対面形式のゼミの回数に関する単峰性という概念を定義し,すべての個人でこれが
成り立つときには,その実施回数の中位によって特徴付けられた選択肢が実現することを証明する.
\ref{subsec:耐戦略性との関連}~節では,ゼミ形式決定手続きと耐戦略性との関連を述べる.特に選択関数の
ドメインを,一般の強選好にした場合と,前節で定義する回数に関する単峰性を満たしたものに制限した場合に
ついて検討する.\ref{subsec:ゼミ形式決定問題の応用}~節では,
この\ref{sec:ゼミ形式決定問題の性質}~節で定義した関数が,ゼミの形式決め以外では
どのような文脈で適応ができるかについて述べる.


\subsection{ゼミ形式決定問題の定式化}\label{subsec:ゼミ形式決定問題の定式化}
この節ではゼミ形式決定問題を社会的選択関数として定式化する.著者のゼミで実際に行われている手続きは,
各回の終了後に次の回の実施形式のみをオンラインか対面かの2択の多数決で決定するというものである.
これから定義される関数も,この,各回ごとに2択の多数決を行うという手続きを
反映しているようなものにする.また便宜上,この社会的選択関数のことを以降「ゼミ形式決定手続き」と
呼ぶことにする.

まずゼミの人数を$n$で表し,さらに$n \geq 2$であるとする.この仮定を置く理由は,分析したい問題が複数人の
意思決定に関するものだからである.またゼミの実施回数を$m$で表し,$m \geq 2$であるとする.これは,$m=1$
だとただの2択の多数決になって面白くないからである.

次に選択肢の集合を定義する.個人が取りうる選択肢は,各回においてオンラインか対面かの2択に分かれるので
$2^m$個存在する.この性質を扱いやすいように選択肢の集合$A$を以下のように定義する.

\begin{definition}[選択肢の集合]
  ゼミ形式決定手続きの選択肢の集合$A$を以下で定義する.
  \[
    A \stackrel{\mathrm{def}}{=} \{00\cdots 00,\ 00\cdots 01,\ 00\cdots 10,\ 00\cdots 11,\
    \ldots,\ 11\cdots 11\}
  \]
  ただし,$A$の要素は$0 \leq a \leq 2^m-1$を満たす$a$を$2$進数で表したものである.さらに$a \in A$
  について,$a$の左から$k$桁目$(k \in \{1,2,\ldots,m \})$の数字が$0$なら$k$回目のゼミをオンライン
  で行い,同様に左から$k$桁目の数字が$1$なら$k$回目のゼミを対面で行うような選択肢を表す.
\end{definition}

\begin{example}
  $m=3$のとき,$A = \{000,\ 001,\ 010,\ 011,\ 100,\ 101,\ 110,\ 111 \}$である.このうち,$101 \in A$は
  $1$回目と$3$回目のゼミを対面形式で実施し,$2$回目のゼミをオンライン形式で実施するような選択肢である.
\end{example}

ゼミ形式決定手続きでは,全$m$回のうち特定の回の実施形式だけに注目したい場合がある.そのために以下の記法を導入する.

\begin{definition}[桁の抽出]
  任意の$a \in A$と,任意の$k \in \{1,2,\ldots,m \}$に対して
  \[
    a(k) \stackrel{\mathrm{def}}{=} a\text{の左から}k\text{桁目の数}
  \]
  とする.
\end{definition}

\begin{example}
  $m = 3$で,$a = 101$とする.このとき$a(1) = 1,\ a(2)=0,\ a(3)=1$である.
\end{example}

また,1回ごとに実施形式を決めていく手続きを念頭に置けば,ある回までの結果次第ではその時点で
実現可能な選択肢は限られる.例えば,$m=3$で1回目の実施形式がオンラインになった場合には,
この時点で実現する可能性の残された選択肢は,1回目にオンライン形式を行うような選択肢,つまり$A$の部分集合
\[
  \{000, 001, 010, 011 \}
\]
の要素だけである.このようにある時点での実現可能な選択肢の集合を表すために
以下の記法を導入する.

\begin{definition}
  任意の$k \in \{1,2,\ldots,m-1\}$と,任意の$d_{1}d_{2}\cdots d_{k}$(ただし$j \in \{1,2,\ldots,k\}$に対して
  $d_j$は$0$または$1$をとる)に対して,選択肢の集合$A$の部分集合$A^{d_{1}d_{2}\cdots d_{k}}$を以下で定める.
  \[
    A^{d_{1}d_{2}\cdots d_{k}} \stackrel{\mathrm{def}}{=} \{a \in A\ |\ 
      \forall i \in \{1,\ldots,k\} \ a(i) = d_i \}
  \]
  とする.
\end{definition}


\begin{example}
  $m=5$とする.このとき,$A^{101}$は$\{10100,\ 10101,\ 10110,\ 10111\}$である.
\end{example}

個人$i$の選好$\succsim_i$について,これ以降では$A$上の強選好$\mathcal{P}$を取ると仮定する.
すると$A$の任意の部分集合に対して,その中で最も好ましい選択肢がただ一つ存在する.
この関係について特に以下のような定義を与える.

\begin{definition}\label{def:最上位選択肢}
  任意の$i \in I$,任意の$k \in \{1,2,\ldots,m-1\}$,さらに任意の$d_{1}d_{2}\cdots d_{k}$
  (ただし$j \in \{1,2,\ldots,k \}$に対して$d_j$は$0$または$1$をとる)に対して$b_{i}$と
  $b_{i}^{d_{1}d_{2}\cdots d_{k}}$を以下のように定める.

  \begin{align*}
    b_i &\stackrel{\mathrm{def}}{=} b \in A\ \text{かつ}\ \forall a \in A \ b \succsim_i a\
    \text{を満たす唯一の}b \\
    b_{i}^{d_{1}d_{2}\cdots d_{k}} &\stackrel{\mathrm{def}}{=}
    b \in A^{d_{1}d_{2}\cdots d_{k}} \ \text{かつ}\ \forall a \in A^{d_{1}d_{2}\cdots d_{k}}
    \ b \succsim_i a\ \text{を満たす唯一の}b
  \end{align*}
\end{definition}

\noindent{}この定義の$b_i$は,選択肢集合$A$の中で個人$i$にとって最も好ましい選択肢を表している.
同様に$b_{i}^{d_{1}d_{2}\cdots d_{k}}$は,$A$の部分集合である$A^{d_{1}d_{2}\cdots d_{k}}$の中で
個人$i$にとって最も好ましい選択肢を表している.

\begin{example}
  $m=3$で,個人$i$の選択肢集合$A$に対する選好$\succsim_i$が
  \[
    101 \succsim_i 011 \succsim_i 110 \succsim_i 010 \succsim_i 001 \succsim_i 100
    \succsim_i 111 \succsim_i 000
  \]
  であるとき,$b_i = 101$,$b_{i}^{0} = 011$,$b_{i}^{00} = 001$である.
\end{example}

以上の定義を用いて,ゼミ形式決定手続きを社会的選択関数として定義することができる.
\begin{definition}[ゼミ形式決定手続き]\label{def:ゼミ形式決定手続き}
  ゼミ形式決定手続きとは,以下の条件を満たす関数$f\colon \mathcal{P}^I \to A$である.\\
  任意の$\succsim \ \in \mathcal{P}^I$に対して$f(\succsim) = d_{1}d_{2}\cdots d_{m} \in A$と定める.
  ただし$k \in \{2,3,\ldots,m\}$として
  \begin{align*}
      d_{1}
    &= \left\{ \begin{array}{@{\,}rl@{}}
      1 & \textup{if}\hspace{5mm}  |\{i \in I\ |\ b_i(1) = 1\}| \hspace{2mm} \geq \hspace{2mm} |\{i \in I\ |\ b_i(1) = 0\}| \\
      0 & \textup{otherwise}
    \end{array} \right. \\
      d_{k}
    &= \left\{ \begin{array}{@{\,}rl@{}}
      1 & \textup{if}\hspace{5mm}  |\{i \in I\ |\ b_i^{d_{1}d_{2}\cdots d_{k-1}}(k) = 1\}| \hspace{2mm}
      \geq \hspace{2mm} |\{i \in I\ |\ b_i^{d_{1}d_{2}\cdots d_{k-1}}(k) = 0\}| \\
      0 & \textup{otherwise}
    \end{array} \right.
  \end{align*}
\end{definition}

このゼミ形式決定手続きの直観的理解は次のようなものである.まず各回の実施形式は,1回目から順番に
オンラインか対面かの2択の多数決によって決められていく.各回では,その時点で実現可能な選択肢のうち
各個人$i$にとって最も好ましい選択肢の情報が使われる.例えば1回目はすべての選択肢が実現可能なので,
この時点で各個人$i$にとって最も好ましい選択肢$b_i$の情報が使われる.ゼミ形式決定手続き$f$は,
1回目の投票であたかも個人$i$が$b_i$を成し遂げるために$b_{i}(1)$に投票したかのようにみなして
対面かオンラインかの多数決を行う.尚,両者の票が同数だった場合は
対面形式\footnote{オンラインにしても以降の議論にはほとんど影響はない.
ただし,必ずどちらか一方を選ぶように決めておく.}にする.$k(\geq 2)$回目では,その時点で実現可能な選択肢の集合は
$A^{d_{1}d_{2}\cdots d_{k-1}}$であるから,この手続きは各個人$i$にとってこの集合の中で最も
好ましい選択肢$b_i^{d_{1}d_{2}\cdots d_{k-1}}$の情報を用いて多数決を行う.つまりゼミ形式決定手続き$f$は,
$k$回目の投票であたかも個人$i$が$b_i^{d_{1}d_{2}\cdots d_{k-1}}$を成し遂げるために
$b_i^{d_{1}d_{2}\cdots d_{k-1}}(k)$に投票したかのようにみなして対面かオンラインかの多数決を行う.

\ref{ボルダルールの欠点}で述べたように,個人に$2^m$個の選択肢に対する選好を表明させることは現実的ではない.
それは今定義したゼミ形式決定手続きに対しても同じことが言える.事実,実際に著者のゼミで行われているゼミの形式決定でも,
一度にすべての回の形式を決めることはせずに,各回ごとにそれぞれが次回の実施形式を2択で投票して多数決で決めているのだった.
しかしここですべての個人が,各回の投票において,その時点で実現可能な選択肢のうちで最も好ましい選択肢が
実現されるように投票を行っているとしたら,その最終的な帰結はゼミ形式決定手続きと全く同じものになる.
この意味において,ゼミ形式決定手続き$f$は実際の手続きとの整合性がある.

\begin{example}
$m=3$かつ$n=3$で,各個人の選好が表~$\ref{tab:3人の選好組}$で与えられたときのゼミ形式決定手続きの値$f(\succsim)$
は次のようにして求められる.まず$f(\succsim) = d_1d_2d_3$とおく.
\[
  b_1 = 101,\ b_2 = 000,\ b_3 = 011
\]
であるから
\[
  b_1(1) = 1,\ b_2(1) = 0,\ b_3(1) = 0
\]
であり,これより$d_1 = 0$となる.次に$d_2$を求める.
\[
  b_1^0 = 011,\ b_2^0 = 000,\ b_3^0 = 010
\]
より
\[
  b_1^0(2) = 1,\ b_2^0(2) = 0,\ b_3^0(2) = 1
\]
が成立し,これより$d_2 = 1$となる.最後に$d_3$を求める.
\[
  b_1^{01} = 011,\ b_2^{01} = 010,\ b_3^{01} = 010
\]
であるから
\[
  b_1^{01}(3) = 1,\ b_2^{01}(3) = 0,\ b_3^{01}(3) = 0
\]
となり,これより$d_3 = 0$となる.したがって$f(\succsim) = 010$である.
  
\end{example}

\begin{table}[h]
  \begin{center}
  \caption{3人の選好組}\label{tab:3人の選好組}
  \begin{tabular}{c|c|c|c} \hline
    & $\succsim_1$ & $\succsim_2$ & $\succsim_3$ \\ \hline
  1位 & 101 & 000 & 010 \\
  2位 & 011 & 100 & 100 \\
  3位 & 110 & 010 & 001 \\
  4位 & 001 & 001 & 101 \\
  5位 & 010 & 110 & 011 \\
  6位 & 100 & 101 & 110 \\
  7位 & 111 & 011 & 111 \\
  8位 & 000 & 111 & 000 \\ \hline 
  \end{tabular}
\end{center}
\end{table}


\subsection{ボルダルールとの比較}\label{subsec:ボルダルールとの比較}
この節では,前節でゼミ形式決定手続きとして定式化した社会的選択関数の性質をボルダルールと比較し,
2つの命題とその証明を記す.

\ref{ボルダルールの利点}節で述べたようにボルダルールには様々な優れた性質であった.その中に
ボルダルールはペア全敗者を選ばないというものがあったが,実はゼミ形式決定手続きにも同じ性質がある.
\begin{proposition}
  ゼミ形式決定手続きはペア全敗者を選ばない.
\end{proposition}
\begin{proof}
  任意の$\succsim$をとり,$f(\succsim) = d_1d_2\cdots d_m$とする.このとき$f(\succsim)$は
  $d_1d_2\cdots d_{m-1}0$か$d_1d_2\cdots d_{m-1}1$のどちらかである.また,
  \begin{equation}\label{eq:い}
    A^{d_1\cdots d_{m-1}} = \{d_1\cdots d_{m-1}0,\ d_1\cdots d_{m-1}1 \}
  \end{equation}
  だから,任意の$i \in I$について$b_i^{d_1\cdots d_{m-1}}$もこの集合の2つの要素のどちらかである.
  したがって,定義~\ref{def:最上位選択肢}より次が成り立つ.
  \begin{equation}\label{eq:あ}
    \begin{gathered}
      b_{i}^{d_1\cdots d_{m-1}}(m) = 1\hspace{5mm} \Leftrightarrow \hspace{5mm} b_{i}^{d_1\cdots d_{m-1}} = d_1\cdots d_{m-1}1 \\
      b_{i}^{d_1\cdots d_{m-1}}(m) = 0\hspace{5mm} \Leftrightarrow \hspace{5mm} b_{i}^{d_1\cdots d_{m-1}} = d_1\cdots d_{m-1}0
    \end{gathered}
  \end{equation}
  
  \noindent$f(\succsim) = d_1\cdots d_{m-1}1$のとき$\colon$定義\ref{def:ゼミ形式決定手続き}より
  \[
    |\{ i \in I \ | \ b_{i}^{d_1\cdots d_{m-1}}(m)=1 \}| \geq |\{ i \in I \ | \ b_{i}^{d_1\cdots d_{m-1}}(m)=0 \}|
  \]
  である.さらに(\ref{eq:あ})より,
  \[
    |\{ i \in I \ | \ b_{i}^{d_1\cdots d_{m-1}}=d_1\cdots d_{m-1}1 \}| \geq
    |\{ i \in I \ | \ b_{i}^{d_1\cdots d_{m-1}}=d_1\cdots d_{m-1}0 \}|
  \]
  が成り立つ.最後に定義~\ref{def:最上位選択肢}と(\ref{eq:い})より,
  \[
    |\{ i \in I \ | \ d_1\cdots d_{m-1}1\ \succsim_i\ d_1\cdots d_{m-1}0 \}| \geq
    |\{ i \in I \ | \ d_1\cdots d_{m-1}0\ \succsim_i\ d_1\cdots d_{m-1}1 \}|
  \]
  したがって,選択肢$d_1\cdots d_{m-1}1$は選択肢$d_1\cdots d_{m-1}0$との1対1の多数決で引き分けるか勝利する.
  ゆえに$d_1\cdots d_{m-1}1$はペア全敗者ではない.

  \noindent$f(\succsim) = d_1\cdots d_{m-1}0$のとき$\colon$定義\ref{def:ゼミ形式決定手続き}より
  \[
    |\{ i \in I \ | \ b_{i}^{d_1\cdots d_{m-1}}(m)=0 \}| > |\{ i \in I \ | \ b_{i}^{d_1\cdots d_{m-1}}(m)=1 \}|
  \]
  である.さらに(\ref{eq:あ})より,
  \[
    |\{ i \in I \ | \ b_{i}^{d_1\cdots d_{m-1}}=d_1\cdots d_{m-1}0 \}| >
    |\{ i \in I \ | \ b_{i}^{d_1\cdots d_{m-1}}=d_1\cdots d_{m-1}1 \}|
  \]
  が成り立つ.最後に定義~\ref{def:最上位選択肢}と(\ref{eq:い})より,
  \[
    |\{ i \in I \ | \ d_1\cdots d_{m-1}0\ \succsim_i\ d_1\cdots d_{m-1}1 \}| >
    |\{ i \in I \ | \ d_1\cdots d_{m-1}1\ \succsim_i\ d_1\cdots d_{m-1}0 \}|
  \]
  したがって,選択肢$d_1\cdots d_{m-1}0$は選択肢$d_1\cdots d_{m-1}1$との1対1の多数決で勝利する.
  ゆえに$d_1\cdots d_{m-1}0$はペア全敗者ではない.
\end{proof}

今証明された命題は,ゼミ形式決定手続きがボルダルールと同じように好ましい性質を持つというものであった.
しかし今度は,ゼミ形式決定手続きがボルダルールとは根本的には違うものであることを示す.

\begin{proposition}\label{pro:ボルダルールとは違う}
  $n$と$m$をそれぞれ$2$以上の任意の自然数とする.このとき,$n$人による$m$回のゼミ形式決定手続きにおいて,
  ボルダ勝者と異なる選択肢を選び取るような選好組$\succsim$が存在する.
\end{proposition}

\begin{proof}
  証明は表~\ref{tab:場合分け}にある煩雑な場合分けによる.以下ではcase$1$からcase$5$まで
  それぞれ証明する.また,選択肢を引数としてそのボルダ得点を返す関数を$p(\cdot)$で表す.
  \begin{table}[h]
    \caption{証明の場合分け表}\label{tab:場合分け}
    \begin{center}
      \begin{tabular}{|c|c|c|c|c|c|}\hline
        \multicolumn{2}{|c|}{} & \multicolumn{4}{c|}{$m$} \\ \cline{3-6}
        \multicolumn{2}{|c|}{} & 2 & 3 & 4 & $\cdots$ \\ \hline
          & 2 & case4 & \multicolumn{3}{c|}{case5} \\ \cline{2-6}
          & 3 & case3 & case2 & \multicolumn{2}{c|}{case1} \\ \cline{2-6}
        $n$ & 4 & case4 & \multicolumn{3}{c|}{case5} \\ \cline{2-6}
          & 5 &  &  &  \multicolumn{2}{c|}{}  \\ \cline{2-2}
          & $\vdots$ & \raisebox{.5\normalbaselineskip}[0pt][0pt]{case3}
            & \raisebox{.5\normalbaselineskip}[0pt][0pt]{case2}
            & \multicolumn{2}{c|}{\raisebox{.5\normalbaselineskip}[0pt][0pt]{case1}} \\ \hline
      \end{tabular}
    \end{center}
  \end{table}

  \noindent{}case$1\colon$$(n = 3$または$n \geq 5)$かつ$m \geq 4$.
  \begin{table}[h]
    \caption{case1での選好}\label{tab:case1}
    \begin{center}
    \begin{tabular}{c|c|c|c}\hline
      & タイプA & タイプB & タイプC \\ \hline
    1位 & $b=1010\cdots$ & $c=0101\cdots$ & $d=1100\cdots$ \\
    2位 & $c=0101\cdots$ & $b=1010\cdots$ & $c=0101\cdots$ \\
    3位 & $\vdots$ & $\vdots$ & $b=1010\cdots$ \\
    $\vdots$ & $\vdots$ & $\vdots$ & $\vdots$ \\ \hline
    \end{tabular}
    \end{center}
  \end{table}
  表~\ref{tab:case1}のような$3$種類の選好を考える.ここで選択肢$b,\ c,\ d$の定義はそれぞれ以下で与えられる.
  \[
    b(i)
    = \left\{ \begin{array}{@{\,}rl@{}}
      0 & \mbox{if}\ i = m \ \text{or} \ i \text{が偶数} \\
      1 & \mbox{otherwise}
    \end{array} \right.
    \hspace{6mm}
    c(i)
    = \left\{ \begin{array}{@{\,}rl@{}}
      0 & \mbox{if}\ i \text{が奇数} \\
      1 & \mbox{otherwise}
    \end{array} \right.
    \hspace{6mm}
    d(i)
    = \left\{ \begin{array}{@{\,}rl@{}}
      0 & \mbox{if}\ i=3,4 \\
      1 & \mbox{if}\ i=1,2 \\
      b_i & \mbox{otherwise}
    \end{array} \right.
  \]
  ここで,さらに$n$の偶奇に注目して場合分けを行う.

  \noindent{}$n$が偶数の場合$\colon$タイプA,B,Cの選好を持つ人数をそれぞれ$\frac{n}{2}-1$,
  $\frac{n}{2}-1$,$2$人とする.このとき,ゼミ形式決定手続きによって実現する選択肢は$b$である.ところが
  $b$と$c$のボルダ得点を計算すると,
  \begin{gather*}
    p(b) = 2^m(\frac{n}{2}-1) + (2^m-1)(\frac{n}{2}-1) + 2(2^m-2) \\
    p(c) = 2^m(\frac{n}{2}-1) + (2^m-1)(\frac{n}{2}-1) + 2(2^m-1)
  \end{gather*}
  であるから選択肢$b$はボルダ勝者ではない.

  \noindent{}$n$が奇数の場合$\colon$タイプA,B,Cの選好を持つ人数をそれぞれ$\frac{n-1}{2}$,
  $\frac{n-1}{2}$,$1$人とする.このときも,ゼミ形式決定手続きによって実現する選択肢は$b$である.
  しかし同様にして,
  \begin{gather*}
    p(b) = 2^m(\frac{n-1}{2}) + (2^m-1)(\frac{n-1}{2}) + 2^m - 2 \\
    p(c) = 2^m(\frac{n-1}{2}) + (2^m-1)(\frac{n-1}{2}) + 2^m - 1
  \end{gather*}
  であるから選択肢$b$はボルダ勝者ではない.

  \noindent{}case$2\colon$$(n = 3$または$n \geq 5)$かつ$m = 3$.
  表~\ref{tab:case2}のような$3$種類の選好を考える.ここでも$n$の偶奇に注目して場合分けを行う.
  \begin{table}[h]
    \caption{case2での選好}\label{tab:case2}
    \begin{center}
      \begin{tabular}{c|c|c|c}\hline
            & タイプA & タイプB & タイプC \\ \hline
        1位 & 101 & 011 & 110 \\ 
        2位 & 011 & 101 & 011 \\ 
        3位 & 110 & 110 & 101 \\ 
        $\vdots$ & $\vdots$ & $\vdots$ & $\vdots$ \\ \hline
      \end{tabular}
    \end{center}
  \end{table}

  \noindent{}$n$が偶数の場合$\colon$タイプA,B,Cの選好を持つ人数をそれぞれ$\frac{n}{2}-1$,
  $\frac{n}{2}-1$,$2$人とする.このとき,ゼミ形式決定手続きによって実現する選択肢は$101$である.
  ところが$101$と$011$のボルダ得点を計算すると,
  \begin{gather*}
    p(101) = 8(\frac{n}{2}-1) + 7(\frac{n}{2}-1) + 6 \\
    p(011) = 8(\frac{n}{2}-1) + 7(\frac{n}{2}-1) + 7
  \end{gather*}
  となるから選択肢$101$はボルダ勝者ではない.

  \noindent{}$n$が奇数の場合$\colon$タイプA,B,Cの選好を持つ人数をそれぞれ$\frac{n-1}{2}$,
  $\frac{n-1}{2}$,$1$人とする.このときも,ゼミ形式決定手続きによって実現する選択肢は$101$であるが,
  $101$と$011$のボルダ得点はそれぞれ
  \begin{gather*}
    p(101) = 8(\frac{n-1}{2}) + 7(\frac{n-1}{2}) + 6 \\
    p(011) = 8(\frac{n-1}{2}) + 7(\frac{n-1}{2}) + 7
  \end{gather*}
  となって選択肢$101$はボルダ勝者ではないことがわかる.

  \noindent{}case$3\colon$$(n = 3$または$n \geq 5)$かつ$m = 2$.
  表~\ref{tab:case3}のような$3$種類の選好を考える.上の場合と同様に$n$の偶奇で場合分けを行う.
  \begin{table}[h]
    \caption{case3での選好}\label{tab:case3}
    \begin{center}
      \begin{tabular}{c|c|c|c}\hline
          & タイプA & タイプB & タイプC \\ \hline
        1位 & 10 & 01 & 11 \\
        2位 & 01 & 10 & 01 \\
        3位 & 00 & 00 & 10 \\
        4位 & 11 & 11 & 00 \\ \hline
      \end{tabular}
    \end{center}
  \end{table}

  


\noindent{}$n$が偶数の場合$\colon$タイプA,B,Cの選好を持つ人数をそれぞれ$\frac{n}{2}-1$,
$\frac{n}{2}-1$,$2$人とする.このとき,ゼミ形式決定手続きによって実現する選択肢は$10$である.
ところが$10$と$01$のボルダ得点を計算すると,
\begin{gather*}
  p(10) = 4(\frac{n}{2}-1) + 3(\frac{n}{2}-1) + 4 \\
  p(01) = 4(\frac{n}{2}-1) + 3(\frac{n}{2}-1) + 6
\end{gather*}
となるから選択肢$10$はボルダ勝者ではない.

\noindent{}$n$が奇数の場合$\colon$タイプA,B,Cの選好を持つ人数をそれぞれ$\frac{n-1}{2}$,
  $\frac{n-1}{2}$,$1$人とする.このときも,ゼミ形式決定手続きによって実現する選択肢は$10$である.
  しかし$10$と$01$のボルダ得点はそれぞれ
  \begin{gather*}
    p(10) = 4(\frac{n-1}{2}) + 3(\frac{n-1}{2}) + 2 \\
    p(01) = 4(\frac{n-1}{2}) + 3(\frac{n-1}{2}) + 3
  \end{gather*}
  となって選択肢$10$はボルダ勝者ではないことがわかる.

  \noindent{}case$4\colon$$(n = 2$または$n = 4)$かつ$m = 2$.
  表~\ref{tab:case4}のような$2$種類の選好を考える.
  \begin{table}[h]
    \caption{case4での選好}\label{tab:case4}
    \begin{center}
      \begin{tabular}{c|c|c}\hline
            & タイプA & タイプB \\ \hline
        1位 & 10 & 11  \\ 
        2位 & 01 & 10  \\ 
        3位 & 00 & 01  \\
        4位 & 11 & 00  \\ \hline
      \end{tabular}
    \end{center}
  \end{table}
  このときタイプA,Bの選好をも持つ人数をそれぞれ$\frac{n}{2}$,$\frac{n}{2}$とすれば,
  ゼミ形式決定手続きによって実現する選択肢は$11$となる.定義~\ref{def:ゼミ形式決定手続き}より,
  各回の形式を決める際に$0$と$1$を支持する人数が同数だった場合には$1$が選ばれるからである.
  しかし一方で
  \begin{gather*}
    p(11) = \frac{5}{2}n \\
    p(10) = \frac{7}{2}n
  \end{gather*}
  となるので,選択肢$11$はボルダ勝者ではない.

  \noindent{}case$5\colon$$(n = 2$または$n = 4)$かつ$m \geq 3$.
  表~\ref{tab:case5}のような$2$種類の選好を考える.ただし選択肢$b,\ c,\ d$の定義は次で
  与えられる.\<\footnote{$3$つの選択肢の$4$桁目以降はここでは任意でよいが,後の議論のために$b,\ c,\ d$ですべて
  同じ値をとることにする.その値をここではすべて$1$にした.}\
  \[
    b(i)
    = \left\{ \begin{array}{@{\,}rl@{}}
      0 & \mbox{if}\ i = 1 \\
      1 & \mbox{otherwise}
    \end{array} \right.
    \hspace{6mm}
    c(i)
    = \left\{ \begin{array}{@{\,}rl@{}}
      0 & \mbox{if}\ i = 3 \\
      1 & \mbox{otherwise}
    \end{array} \right.
    \hspace{6mm}
    d(i)
    = \left\{ \begin{array}{@{\,}rl@{}}
      0 & \mbox{if}\ i=2 \\
      1 & \mbox{otherwise}
    \end{array} \right.
  \]
  
  \begin{table}[h]
    \caption{case5での選好}\label{tab:case5}
    \begin{center}
      \begin{tabular}{c|c|c}\hline
          & タイプA & タイプB \\ \hline
        1位 & $b = 011\cdots$ & $d = 101\cdots$ \\
        2位 & $c = 110\cdots$ & $b = 011\cdots$ \\
        3位 & $d = 101\cdots$ & $c = 110\cdots$ \\
        $\vdots$ & $\vdots$ & $\vdots$ \\ \hline
      \end{tabular}
    \end{center}
  \end{table}

  \noindent{}ここでタイプA,Bの選好を持つ人数をどちらも$\frac{n}{2}$人とする.このとき,ゼミ形式決定手続きによって
  選ばれる選択肢は$c$となる.しかしボルダ得点を計算すると
  \begin{gather*}
    p(b) = \frac{n}{2}(2^{m+1}-1) \\
    p(c) = \frac{n}{2}(2^{m+1}-3)
  \end{gather*}
  となるので,$c$はボルダ勝者にはならない.以上ですべての場合の証明が終了した.
\end{proof}

\subsection{単峰性との関連}\label{subsec:単峰性との関連}
  この節では,\ref{subsec:単峰性と中位選択関数}節で紹介した単峰性に関する議論をゼミ形式決定手続きに応用する.
  具体的には,単峰性の概念を直和に分解された選択肢の集合上に拡張し,その拡張された概念に特徴付けられた
  選択肢がゼミ形式決定手続きによって選ばれることを証明する.

  まず例として次のような状況を考える.あるゼミのメンバーは,選択肢の良し悪しに関する第一の基準として
  対面形式の実施回数を気にしているとする.例えばこの人は対面形式を$k$回行う選択肢を最も高く順序づけており,対面
  形式の実施回数がそこから多い方であれ少ない方であれ離れるほど,より好ましくなくなっていく.このとき,選択肢全体を
  対面形式を何回行うかによってグループ分けすると,そのグループ同士の上に単峰性が成り立っていることになる.この例のような
  議論を扱うために,以降いくつかの定義を導入する.

  先の例では,ある選択肢が対面形式のゼミを何回実施するのかということに注目した.まずはこれを表す記法を定義する.
  \begin{definition}
    選択肢$a \in A$に対して$\mathrm{pop}(a)$を次で定義する.
    \[
      \mathrm{pop}(a) \stackrel{\mathrm{def}}{=} |\{k \in \{1, 2, \ldots, m \} \ | \
      a(k) = 1 \}|
    \]
  \end{definition}

  \begin{example}
    $m=5$とする.このとき,$10101,\ 11000,\ 00000 \in A$に対して,
    \[
      \mathrm{pop}(10101) = 3, \hspace{5mm} \mathrm{pop}(11000) = 2, \hspace{5mm} \mathrm{pop}(00000) = 0
    \]
    である.
  \end{example}

  各選択肢の対面実施回数によって何をしたかったかといえば,その回数ごとに選択肢全体のグループ分けをすることである.
  そのグループは今定義した$\mathrm{pop(\cdot)}$を使って次のように容易に定義できる.

  \begin{definition}
    $k(0 \leq k \leq m)$に対して
    \[
      S_k \stackrel{\mathrm{def}}{=} \{a \in A\ | \ \mathrm{pop}(a) = k \}
    \]
    と定める.$S_k$は対面形式を$k$回行うような選択肢全体の集合である.
  \end{definition}

  \begin{example}
    $m = 3$とする.このとき,
    \begin{align*}
      S_0 =& \{ 000 \} \\
      S_1 =& \{ 001,\ 010,\ 100 \} \\
      S_2 =& \{ 011,\ 101,\ 110 \} \\
      S_3 =& \{ 111 \}
    \end{align*}
    である.
  \end{example}

  以上を用いて,対面形式の実施回数に注目したときの単峰性を定義することができる.ここでは特にこれを
  回数に関する単峰性と呼ぶ.
  \begin{definition}[回数に関する単峰性]\label{def:回数に関する単峰性}
    個人$i$の選好$\succsim_i$が回数に関する単峰性を満たすとは,ベストな回数$k(\succsim_i) \in \{0,1,\ldots, m\}$がただ一つ存在して,
    任意の$l,l' \in \{0,1,\ldots, m\}$に対して
    \[
      k(\succsim_i) \neq l \Rightarrow \forall b \in S_{k(\succsim_i)} \ \forall a \in S_l \ b \succ_i a
    \]
    かつ
    \[
      [\mathnormal{l}' < \mathnormal{l} < k(\succsim_i) \text{または} k(\succsim_i) < \mathnormal{l} < \mathnormal{l}'] \Rightarrow
      \forall a' \in S_{\mathnormal{l}'} \ \ \forall a \in S_{\mathnormal{l}} \ \ \forall b \in S_{k(\succsim_i)}
      \ \ b(\succsim_i) \succ_i a \succ_i a'
    \]
    が成り立つことである.
  \end{definition}

  \begin{table}[h]
    \caption{回数に関する単峰性を満たした選好}\label{tab:回数に関する単峰性}
    \begin{center}
      \begin{tabular}{c|cccccccc}
          & 1位 & 2位 & 3位 & 4位 & 5位 & 6位 & 7位 & 8位 \\ \hline
        $\succsim_1$ & 101 & 011 & 110 & 100 & 001 & 010 & 111 & 000 \\ \hline
        $\succsim_2$ & 000 & 100 & 010 & 001 & 011 & 101 & 110 & 111 \\
      \end{tabular}
    \end{center}
  \end{table}

\begin{example}
  $m=3$とする.このとき,表~$\ref{tab:回数に関する単峰性}$における$\succsim_1$と$\succsim_2$は
  共に回数に関する単峰性を満たしている.また$k(\succsim_1) = 2$,$k(\succsim_2) = 0$である.
\end{example}

さらに,すべての個人の選好が回数に関する単峰性を満たすときには,各個人にとってベストな回数$k(\succsim_i) \in \{0,1,\ldots,m\}$が
存在するので,それらの中位を定義することができる.ここでは特にそれを中位回数として定義する.

\begin{definition}[中位回数]\label{def:中位回数}
  すべての個人の選好が回数に関する単峰性を満たすとき,各$i \in I$にとってベストな回数$k(\succsim_1),\ldots,k(\succsim_n)$
  が存在する.これらの中位を中位回数と呼び,$k^m(\succsim)$と書く.
  すなわち$k^m(\succsim) \in \{k(\succsim_1), \ldots, k(\succsim_n) \}$かつ
  \begin{gather*}
    |\{i \in I \ | \ k(\succsim_i) \leq k^m(\succsim) \}| \ \geq \ \frac{n}{2} \\
    |\{i \in I \ | \ k(\succsim_i) \geq k^m(\succsim) \}| \ \geq \ \frac{n}{2}
  \end{gather*}
  を満たす.$n$が偶数のときに中位が$2$つ存在する場合があるが,その場合は大きい方\footnote{
    定義\ref{def:ゼミ形式決定手続き}の各$d_i$を決める箇所において,オンラインと対面が同数だったときに対面ではなく
    オンラインを採用するならば,ここの中位回数は小さい方にする.
  }
  のみを中位回数とする.
\end{definition}

この回数に関する単峰性の議論をまとめる.すべての個人の選好が回数に関する単峰性を満たすとき,つまり,
各個人には最も好ましい対面ゼミの実施回数があり,そこから離れるほどより好ましくなくなるような選好を
持っているとき,ゼミ形式決定手続きが選び取る選択肢は,対面形式をちょうど中位回数だけ行うような選択肢になる.

\begin{proposition}\label{prop:中位回数選択命題}
  $f$をゼミ形式決定手続きとし,すべての個人の選好が回数に関する単峰性を満たすとする.
  このとき中位回数$k^m(\succsim)$に対して
  \[
    f(\succsim) \in S_{k^m(\succsim)}
  \]
  が成り立つ.
\end{proposition}

\begin{proof}
  $f(\succsim) = d_{1}d_{2}\cdots d_{m}$とする.まず,$n$の偶奇で場合わけを行う.\\
  $n$が奇数のとき$\colon$中位回数を$k^m(\succsim)$をおく.このとき中位回数の定義~$\ref{def:中位回数}$より
  以下が成り立つ.
  \begin{gather}
    |\{ i \in I \ | \ k(\succsim_i) \leq k^{m}(\succsim) \}|  \geq \frac{n+1}{2} \label{eq:ああ}\\ 
    |\{ i \in I \ | \ k(\succsim_i) \geq k^{m}(\succsim) \}|  \geq \frac{n+1}{2} \label{eq:いい}
  \end{gather}
  ここで,ある$l \in \{1,2,\ldots,m\}$が存在して,
  \begin{equation}\label{eq:うう}
    |\{ j \in \{1,2,\ldots,l-1\} | d_j = 1 \}| = k^{m}(\succsim)
  \end{equation}
  が成り立つときの$d_l$について考える.\<\footnote{$l=1$のとき集合$\{1,2,\ldots,l-1\}$は空集合である.}\
  つまりこれは$l-1$回目までに対面形式のゼミを$k^{m}(\succsim)$回
  行っているような状況を考えている.これから$d_l=0$を示す.まず$(\ref{eq:うう})$より
  \begin{equation}\label{eq:かか}
    f(\succsim) = d_{1}d_{2}\cdots d_{m} \in \bigcup_{j=k^{m}(\succsim)}^{m}S_j
  \end{equation}
  であることに注意する.これは$d_{1}d_{2}\cdots d_{m}$に含まれる$1$の数は
  $k^{m}(\succsim)$個以上であることを表している.ここで
  \mbox{$i \in \{j \in I \ | \ k(\succsim_j) \leq k^{m}(\succsim) \}$}
  である個人$i$について考えると,その選好の回数に関する単峰性より
  \[
    \forall l \in \{k^m(\succsim)+1,k^m(\succsim)+2, \ldots, m\} \ \forall b \in S_{k^m(\succsim)}
    \ \forall a \in S_l \ b \succ_i a
  \]
  が成立する.したがって$(\ref{eq:かか})$を踏まえると,$d_l$を決める時点では,個人$i$にとって最も好ましい選択肢は
  対面形式ゼミを$k^m(\succsim)$回行う選択肢である.すなわち
  \[
    b_{i}^{d_{1}\cdots d_{l-1}} \in S_{k^m(\succsim)}
  \]
  である.これと$(\ref{eq:うう})$より
  \[
    b_{i}^{d_{1}\cdots d_{l-1}}(l) = 0
  \]
  となる.$(\ref{eq:ああ})$より,このような個人$i$は$\frac{n+1}{2}$人以上いるので
  \[
    |\{i \in I \ | \ b_{i}^{d_{1}\cdots d_{l-1}}(l) = 0 \}| >
    |\{i \in I \ | \ b_{i}^{d_{1}\cdots d_{l-1}}(l) = 1 \}|
  \]
  である.ゆえに$d_l=0$となる.以上の推論で示されたことは次のことである.\<\footnote{同じことであるが,
  $\mathrm{pop}(f(\succsim)) \leq k^m(\succsim)$としてもよい.こちらの方が多少
  目に優しいかもしれない.}\
  \begin{equation}\label{eq:A}
    \forall l \in \{k^m(\succsim)+1,k^m(\succsim)+2, \ldots, m\} \ f(\succsim) = d_{1}d_{2}\cdots d_{m} \notin S_l
  \end{equation}
  これは,対面形式のゼミを$k^m(\succsim)$回より多く行うような選択肢はゼミ形式決定手続きには選ばれない
  ということを表している.\\
  次にある$l \in \{1,2,\ldots,m\}$が存在して,
  \begin{equation}\label{eq:ええ}
    |\{ j \in \{1,2,\ldots l-1\} \ | \ d_j=1 \}| + m + 1 - l = k^m(\succsim)
  \end{equation}
  となるときの$d_l$について考える.これはつまり,このあと$d_l$から$d_m$まですべてが$1$にならないと
  $\mathrm{pop}(f(\succsim)) < k^m(\succsim)$となってしまう状況である.これから$d_l=1$を示す.
  まず$(\ref{eq:ええ})$より$\mathrm{pop}(f(\succsim))$は高々$k^m(\succsim)$なので
  \begin{equation}\label{eq:おお}
    f(\succsim) \in \bigcup_{j=0}^{k^m(\succsim)}S_j
  \end{equation}
  である.ここで,$i \in \{j \in I \ | \ k(\succsim_j) \geq k^m(\succsim) \}$となる個人$i$を考えると,
  その選好の回数に関する単峰性より
  \begin{equation*}
    \forall l \in \{0,1,\ldots, k^m(\succsim)-1\}\ \forall b \in S_{k^m(\succsim)} \ 
    \forall a \in S_l \ b \succ_i a
  \end{equation*}
  が成り立つ.したがって$(\ref{eq:おお})$を踏まえると
  \begin{equation*}
    b_i^{d_{1}\cdots d_{l-1}} \in S_{k^m(\succsim)}
  \end{equation*}
  である.これと$(\ref{eq:ええ})$より
  \begin{equation*}
    b_i^{d_{1}\cdots d_{l-1}}(l) = 1
  \end{equation*}
  である.$(\ref{eq:いい})$より,このような個人は$\frac{n+1}{2}$人以上いるので
  \begin{equation*}
    |\{i \in I \ | \ b_{i}^{d_{1}\cdots d_{l-1}}(l) = 1 \}| >
    |\{i \in I \ | \ b_{i}^{d_{1}\cdots d_{l-1}}(l) = 0 \}|
  \end{equation*}
  が成り立つことになる.したがって定義~\ref{def:ゼミ形式決定手続き}より$d_l=1$となる.
  以上の推論によって次が示されたことになる.
  \begin{equation}\label{eq:B}
    \forall l \in \{0,1,\ldots, k^m(\succsim)-1\} \ f(\succsim) \notin S_l
  \end{equation}
  これは,対面形式のゼミを$k^m(\succsim)$回より少なく行うような選択肢はゼミ形式決定手続きには
  選ばれないということを表している.\\
  したがって$(\ref{eq:A})$と$(\ref{eq:B})$より$f(\succsim) \in S_{k^m(\succsim)}$が示された.\\
  $n$が偶数のとき$\colon$ここではさらに各$k(\succsim_i)$の中位が$1$つのときと$2$つのときで場合わけを
  行う.\<\footnote{紛らわしいので注意が必要だが,定義~\ref{def:中位回数}より,中位が$2$つ
  ある場合でも中位回数は$1$つだけである.}\\
  中位が$1$つのとき$\colon$まず
  \begin{gather}
    |\{ i \in I \ | \ k(\succsim_i) \leq k^m(\succsim) \}| \geq \frac{n}{2}+1 \label{eq:あああ} \\
    |\{ i \in I \ | \ k(\succsim_i) \geq k^m(\succsim) \}| \geq \frac{n}{2}+1 \label{eq:いいい}
  \end{gather}
  であることに注意する.なぜならば,$k^m(\succsim)$は唯一の中位なので,特に
  \mbox{$\{k(\succsim_1),k(\succsim_2),\ldots,k(\succsim_n)\}$}の中で,$k^m(\succsim)$の
  次に大きい$k(\succsim_j)$は中位ではない.したがって中位の定義を否定した
  \begin{equation*}
    |\{i \in I \ | \ k(\succsim_i) \leq k(\succsim_j) \}| < \frac{n}{2} \hspace{5mm} \text{または} 
    \hspace{5mm} |\{i \in I \ | \ k(\succsim_i) \geq k(\succsim_j) \}| < \frac{n}{2}
  \end{equation*}
  が成り立つが,$k^m(\succsim) < k(\succsim_j)$を踏まえれば,真であるのは後者の
  \begin{equation}\label{eq:ううう}
    |\{i \in I \ | \ k(\succsim_i) \geq k(\succsim_j) \}| < \frac{n}{2}
  \end{equation}
  である.$k(\succsim_j)$が\mbox{$\{k(\succsim_1),k(\succsim_2),\ldots,k(\succsim_n)\}$}の中で,
  $k^m(\succsim)$の次に大きいことを踏まえると
  \begin{equation*}
    |\{i \in I \ | \ k(\succsim_i) \leq k^m(\succsim) \}| +
    |\{i \in I \ | \ k(\succsim_i) \geq k(\succsim_j) \}| = n
  \end{equation*}
  だから,これと$(\ref{eq:ううう})$より$(\ref{eq:あああ})$が成り立つ.
  同様にして$(\ref{eq:いいい})$が成り立つことも示される.これからこの$2$つの事実を用いて
  元の命題の証明を進める.\\
  まず,ある$l \in {\{1,2,\ldots,m\}}$が存在して
  \begin{equation}\label{eq:えええ}
    |\{ j \in \{1,2,\ldots,l-1 \} \ | \ d_j = 1| = k^m(\succsim)
  \end{equation}
  が成り立っているときの$d_l$について考える.これは$l-1$回目までに対面形式のゼミをちょうど$k^m(\succsim)$
  回行うような場合である.このとき$d_l=0$になることを示す.まず$(\ref{eq:えええ})$より
  \begin{equation}\label{eq:おおお}
    f(\succsim) \in \bigcup_{j=k^m(\succsim)}^{m}S_j
  \end{equation}
  である.このとき$i \in \{ j \in I \ | \ k(\succsim_j) \leq k^m(\succsim)\}$である個人$i$に
  注目すると,その選好の回数に関する単峰性より
  \begin{equation*}
    \forall l \in \{k^m(\succsim)+1, \ldots, m\} \ \forall b \in S_{k^m(\succsim)} \ 
    \forall a \in S_l \ b \succ_i a
  \end{equation*}
  が成り立ち,したがって$(\ref{eq:おおお})$を踏まえると
  \begin{equation*}
    b_{i}^{d_1 \cdots d_{l-1}} \in S_{k^m(\succsim)}
  \end{equation*}
  がいえる.これと$(\ref{eq:えええ})$より,
  \begin{equation*}
    b_{i}^{d_1 \cdots d_{l-1}}(l) = 0
  \end{equation*}
  が成立する.そして$(\ref{eq:あああ})$よりこのような個人$i$は$\frac{n}{2}+1$人以上いるので,
  \begin{equation*}
    |\{i \in I \ | \ b_i^{d_1 \cdots c_{l-1}}(l) = 0\}| >
    |\{i \in I \ | \ b_i^{d_1 \cdots c_{l-1}}(l) = 1\}|
  \end{equation*}
  となる.このことから$d_l=0$がしたがう.以上の推論によって次が示されたことになる.
  \begin{equation}\label{eq:ききき}
    \forall l \in \{k^m(\succsim)+1, \ldots, m\} \ f(\succsim) \notin S_l
  \end{equation}
  次に,ある$l \in \{1,2, \ldots, m\}$が存在して
  \begin{equation}\label{eq:かかか}
    |\{j \in \{1,2, \ldots, l-1\} \ | \ d_j=1 \}| + m + 1 -  l = k^m(\succsim)
  \end{equation}
  が成り立つときの$d_l$について考える.これは,$d_l$から$d_m$まですべて$1$にならなければ
  $\mathrm{pop}(f\succsim) < k^m(\succsim)$となる状況である.このとき$d_l=1$になることを
  示す.まず$(\ref{eq:かかか})$より
  \begin{equation}\label{eq:けけけ}
    f(\succsim) \in \bigcup_{j=0}^{k^m(\succsim)} S_j
  \end{equation}
  が成り立つ.ここで$i \in \{ j \in I \ | \ k(\succsim_j) \geq k^m(\succsim) \}$となる
  個人$i$に注目すると,回数に関する単峰性より
  \begin{equation*}
    \forall l \in \{0,1, \ldots, k^m(\succsim)-1\} \ \forall b \in S_{k^m(\succsim)} \ 
    \forall a \in S_l \ b \succ_i a
  \end{equation*}
  が成り立つ.したがって$(\ref{eq:けけけ})$より
  \begin{equation*}
    b_i^{d_1 \cdots d_{l-1}} \in S_{k^m(\succsim)}
  \end{equation*}
  が成り立ち,これと$(\ref{eq:かかか})$より
  \begin{equation*}
    b_i^{d_1 \cdots d_{l-1}}(l) = 1
  \end{equation*}
  となる.そして$(\ref{eq:いいい})$よりこのような個人$i$は$\frac{n}{2}+1$人以上いるので
  \begin{equation*}
    |\{i \in I \ | \ b_i^{d_1 \cdots d_{l-1}}(l) = 1 \}| >
    |\{i \in I \ | \ b_i^{d_1 \cdots d_{l-1}}(l) = 0 \}|
  \end{equation*}
  となり,したがって$d_l=1$である.以上の推論より次が示された.
  \begin{equation}\label{eq:くくく}
    \forall l \in \{0,1, \ldots, k^m(\succsim)\} \ f(\succsim) \notin S_l
  \end{equation}
  したがって$(\ref{eq:ききき})$と$(\ref{eq:くくく})$より$f(\succsim) \in S_{k^m(\succsim)}$
  が証明された.\\
  中位が$2$つのとき$\colon$$2$つある中位のうち,小さい方を$kl^m(\succsim)$,大きい方を$kr^m(\succsim)$
  とする.\<\footnote{数直線上に並べたときに小さい方がleft,大きい方がrightにあるのでその頭文字を使った.}\
  中位回数の定義より,このうち中位回数は$kr^m(\succsim)$である.したがってこれから$f(\succsim) \in
  S_{kr^m(\succsim)}$を示していくことになる.まず中位の定義から$kl^m(\succsim),kr^m(\succsim) \in
  \{k(\succsim_1),\ldots,k(\succsim_n)\}$かつ
  \begin{gather}
    |\{i \in I \ | \ k(\succsim_i) \leq kl^m(\succsim) \}| \geq \frac{n}{2} \notag \\
    |\{i \in I \ | \ k(\succsim_i) \geq kl^m(\succsim) \}| \geq \frac{n}{2} \label{eq:ててて} \\
    |\{i \in I \ | \ k(\succsim_i) \leq kr^m(\succsim) \}| \geq \frac{n}{2} \label{eq:さささ} \\
    |\{i \in I \ | \ k(\succsim_i) \geq kr^m(\succsim) \}| \geq \frac{n}{2} \notag
  \end{gather}
  が成り立っている.ここでさらに
  \begin{equation}\label{eq:ししし}
    |\{i \in I \ | \ k(\succsim_i) \leq kr^m(\succsim) \}| \geq \frac{n}{2}+1
  \end{equation}
  も成り立つことがわかる.なぜならば$(\ref{eq:さささ})$,$kr^m(\succsim) \in \{k(\succsim_1), \ldots,
  k(\succsim_n)\}$,そして$kl^m(\succsim) < kr^m(\succsim)$より
  \begin{eqnarray*}
    |\{i \in I \ | \ k(\succsim_i) \leq kr^m(\succsim) \}| \hspace{5mm} \geq & \hspace{5mm}
    |\{i \in I \ | \ k(\succsim_i) \leq kl^m(\succsim) \}| + 1 \\
    \geq & \frac{n}{2}+1
  \end{eqnarray*}
  となるからである.この事実を用いて中位が$2$つある場合の証明を行う.\\
  まずある$l \in \{1,2,\ldots,m\}$が存在して
  \begin{equation}\label{eq:すすす}
    |\{j \in \{1,2,\ldots, l-1\} \ | \ d_j=1 \}| = kr^m(\succsim)
  \end{equation}
  が成り立つときの$d_l$について考える.これから$d_l=0$を証明する.最初に$(\ref{eq:すすす})$より
  \begin{equation}\label{eq:せせせ}
    f(\succsim) \in \bigcup_{j=kr^m(\succsim)}^m S_j
  \end{equation}
  である.ここで$i \in \{j \in I \ | \ k(\succsim_j) \leq kr^m(\succsim) \}$となる個人$i$に
  注目すると,その選好の回数に関する単峰性より
  \begin{equation}\label{eq:そそそ}
    \forall l \in \{kr^m(\succsim) + 1,kr^m(\succsim)+2, \ldots, m\} \ \forall b \in S_{kr^m(\succsim)} \ 
    \forall a \in S_l \ b \succ_i a
  \end{equation}
  が成り立つ.したがって
  \begin{equation*}
    b_i^{d_1 \cdots d_{l-1}} \in S_{kr^m(\succsim)}
  \end{equation*}
  となり,これと$(\ref{eq:すすす})$より
  \begin{equation*}
    b_i^{d_1 \cdots d_{l-1}}(l) = 0
  \end{equation*}
  が成り立つ.そして$(\ref{eq:ししし})$よりこのような個人$i$は$\frac{n}{2}+1$人以上いるので
  \begin{equation}
    |\{ i \in I \ | \ b_i^{d_1 \cdots d_{l-1}}(l) = 0 \}| >
    |\{ i \in I \ | \ b_i^{d_1 \cdots d_{l-1}}(l) = 1 \}|
  \end{equation}
  となり,したがって$d_l=0$が成り立つ.以上の推論より次が得られる.
  \begin{equation}\label{eq:たたた}
    \forall l \in \{kr^m(\succsim)+1,kr^m(\succsim)+2, \ldots, m\} \ f(\succsim) \notin S_l
  \end{equation}
  次に,ある$l \in \{1,2,\ldots, m\}$が存在して
  \begin{equation}\label{eq:ちちち}
    |\{j \in \{1,2,\ldots,l-1\} \ | \ d_j=1\}| + m + 1 - l = kr^m(\succsim)
  \end{equation}
  が成り立っているときの$d_l$について考える.まず$(\ref{eq:ちちち})$より
  \begin{equation}\label{eq:つつつ}
    f(\succsim) \in \bigcup_{j=0}^{kr^m(\succsim)} S_j
  \end{equation}
  である.ここで$i \in \{j \in I \ | \ k(\succsim_j) \geq kr^m(\succsim)\}$となる個人$i$に
  注目すると,その選好の回数に関する単峰性より
  \begin{equation}
    \forall l \in \{0,1,\ldots, kr^m(\succsim)-1\} \ \forall b \in S_{kr^m(\succsim)} \ 
    \forall a \in S_l \ b \succ_i a
  \end{equation}
  が成立する.したがって$(\ref{eq:つつつ})$より
  \begin{equation*}
    b_i^{d_1 \cdots d_{l-1}} \in S_{kr^m(\succsim)}
  \end{equation*}
  となる.これと$(\ref{eq:ちちち})$より
  \begin{equation*}
    b_i^{d_1 \cdots d_{l-1}}(l) = 1
  \end{equation*}
  が成り立つ.そして$(\ref{eq:ててて})$より,このような個人$i$は$\frac{n}{2}$人以上いるので
  \begin{equation*}
    |\{ i \in I \ | \ b_i^{d_1 \cdots d_{l-1}}(l) = 1 \}| \geq
    |\{ i \in I \ | \ b_i^{d_1 \cdots d_{l-1}}(l) = 0 \}|
  \end{equation*}
  となる.ゆえに$d_l=1$である.以上の推論より次の結果が得られた.
  \begin{equation}\label{eq:ととと}
    \forall l \in \{0,1,\ldots, kr^m(\succsim)-1\} \ f(\succsim) \notin S_l
  \end{equation}
  最後に$(\ref{eq:たたた})$と$(\ref{eq:ととと})$より$f(\succsim) \in S_{kr^m(\succsim)}$が
  導かれる.\\
  以上ですべての場合の証明が終了した.
\end{proof}

また,命題~\ref{pro:ボルダルールとは違う}ではゼミ形式決定手続きがボルダルールと異なる選択肢を選び取る
選好組があることを証明したが,これはすべての個人の選好が回数に関する単峰性を満たしていても同様のことが成り立つ.
これは命題~\ref{pro:ボルダルールとは違う}の証明に使われた選好組が,既に回数に関する単峰性を満たしている,または
満たすように明記されていない部分を補足することができることから明らかである.



\subsection{耐戦略性との関連}\label{subsec:耐戦略性との関連}
  この節では,\ref{sec:戦略的操作の可能性}~節で扱った耐戦略性に関する議論と,ゼミ形式決定
  手続きとの関連について述べる.具体的には,ゼミ形式決定手続きにおけるドメインを,一般的な強選好と
  前節で定義した回数に関する単峰性を満たした選好の2つの場合について,それぞれ耐戦略性を満たすか
  検討する.

  まず一般的な強選好ドメインについては,ギバード=サタスウェイト定理
  (定理~\ref{thm:ギバード=サタスウェイト定理})を利用することで,耐戦略性を満たさないことが
  比較的容易にわかる.

  \begin{proposition}
    任意の$i \in I$に対して$\mathcal{D}_i = \mathcal{P}$とする.このとき,ゼミ形式決定手続き
    $f\colon \mathcal{D}_I \to A$は耐戦略性を満たさない.
  \end{proposition}
    
  \begin{proof}
    ギバード=サタスウェイト定理(定理~\ref{thm:ギバード=サタスウェイト定理})より,
    $f$が全射性を満たすことと,独裁制ではないことを示せば十分である.

    まず$f$が全射性であることを示す.$a \in A$を任意にとる.このとき,任意の$i \in I$に対して
    $b_i = a$,つまりすべての個人にとって$a$が最も好ましいような選好組$\succsim \ \in \mathcal{D}_I$
    に対して$f(\succsim) = a$が成り立つ.故に$f$は全射性を満たす.

    次に$f$が独裁制ではないことを示す.そのために,独裁制の定義(定義~\ref{def:独裁制})を否定した
    \begin{equation*}
      \forall i \in I \hspace{5mm} \exists \succsim \ \in \mathcal{D}_I \hspace{5mm} 
      \exists x \in A \hspace{5mm} f(\succsim) \prec_i x
    \end{equation*}
    を示す.まず$i \in I$を任意にとり,$\succsim_i$は$b_i(1)=0$を満たすようなものとする.
    次に$\succsim_j\ (j \neq i)$に関しては$b_j(1) = 1$を満たすようなものとする.
    このとき明らかに$f(\succsim)(1) = 1$であるが,$b_i(1)=0$より,
    $f(\succsim) \prec_i b_i$である.したがって$i$は独裁者ではない.この$i$は任意であったので,
    $f$は独裁制ではない.
  \end{proof}

  次に,ゼミ形式決定手続きのドメインを,回数に関する単峰性を満たす選好に限定した場合の耐戦略性について
  検討する.その前にまず,「回数に関する耐戦略性」とでも言えるような,次の性質が成り立つことが示される.

  \begin{proposition}
    すべての$i \in I$に対して,$\mathcal{D}_i$は回数に関する単峰性を満たした選好全体の集合とする.
    このとき次が成り立つ.
    \begin{equation*}
      \forall i \in I \hspace{4mm} \forall \succsim \ \in \mathcal{D}_I \hspace{4mm}
      \forall \succsim'_i \ \in \mathcal{D}_i \hspace{4mm}
      [\mathrm{pop}(f(\succsim)) \neq \mathrm{pop}(f(\succsim'_i, \succsim_{-i}))]
      \Rightarrow [f(\succsim) \succ_i f(\succsim'_i, \succsim_{-i})]
    \end{equation*}
  \end{proposition}

  証明の前に,この命題の述べるところを説明しておく.これはつまり,虚偽の選好を表明することで
  対面形式の実施回数を変更することができたとしても,自分にとって悪い方にしか操作ができない
  ということである.嘘をついても回数のレベルでは得ができないので,この意味で「回数に関する耐戦略性」
  と言えるような性質が成り立っているということである.

  \begin{proof}
    $i \in I$,$\succsim \ \in \mathcal{D}_I$,$\succsim'_i \in \ \mathcal{D}_i$を任意にとる.
    すべての個人の選好は回数に関する単峰性を満たすので,$\succsim$に対して中位回数$k^m(\succsim)$が
    ただ一つ決まる.そして命題~\ref{prop:中位回数選択命題}より
    \begin{equation*}
      \mathrm{pop}(f(\succsim)) = k^m(\succsim)
    \end{equation*}
    が成り立つことに注意する.以上を踏まえて$i$に関する場合わけを行う.

    $k(\succsim_i) = k^m(\succsim)$であるときは,回数に関する単峰性の定義より直ちに成り立つ.

    $k(\succsim_i) < k^m(\succsim)$であるときは,$i$が虚偽の選好$\succsim'_i$を表明して
    新たな中位回数$k^m(\succsim'_i, \succsim_{-i})$を実現できたとしても,中位回数の定義より
    $k^m(\succsim) < k^m(\succsim'_i, \succsim_{-i})$にしかならない.回数に関する単峰性
    の定義より,これは$i$にとってより好ましくない回数である.

    $k^m(\succsim) < k(\succsim_i)$であるときも同様に示される.
  \end{proof}

  では,普通の耐戦略性についてはどうだろうか.これについてはいかなる場合も成り立たないと予想する.
  つまり,$n$と$m$が2以上のどんな値でも,虚偽の選好を表明することで得をする個人と選好組が存在する.
  本論文ではその完全な証明を与えるには至らなかったが,証明の一部として反例を1つあげる.

  \begin{proposition}
    $n$,$m$を$2$以上の任意の自然数とし,すべての$i \in I$に対して,
    $\mathcal{D}_i$は回数に関する単峰性を満たした選好全体の集合とする.
    このとき,ゼミ形式決定手続き$f\colon \mathcal{D}_I \to A$は耐戦略性を満たさない.
  \end{proposition}

  \begin{example}[$n=3,\ m=3$の場合の反例]
    表~$\ref{tab:耐戦略性への反例}$のような$3$人の選好組を考える.
    この$4$つの選好はいずれも回数に関する単峰性を満たしているが
    \begin{equation*}
      f(\succsim_1, \succsim_2, \succsim_3) = 011
      \prec_3
      110 = f(\succsim_1, \succsim_2, \succsim'_3) 
    \end{equation*}
    が成り立つので,$f$は耐戦略性を満たしていない.
    \begin{table}[h]
      \caption{耐戦略性への反例}\label{tab:耐戦略性への反例}
      \begin{center}
        \begin{tabular}{c|c|c|c|c}\hline
            & $\succsim_1$ & $\succsim_2$ & $\succsim_3$ & $\succsim'_3$\\ \hline
          1位 & $101$ & $011$ & $000$ & $110$ \\
          2位 & $011$ & $110$ & $001$ & $101$ \\
          3位 & $110$ & $101$ & $100$ & $011$ \\
          4位 & $100$ & $111$ & $010$ & $111$ \\
          5位 & $010$ & $001$ & $110$ & $100$ \\
          6位 & $001$ & $010$ & $101$ & $010$ \\
          7位 & $111$ & $100$ & $011$ & $001$ \\
          8位 & $000$ & $000$ & $111$ & $000$ \\ \hline
        \end{tabular}
      \end{center}
    \end{table}
  \end{example}


\subsection{ゼミ形式決定問題の応用}\label{subsec:ゼミ形式決定問題の応用}
  この\ref{sec:ゼミ形式決定問題の性質}~節では,ゼミ形式決定問題という名前で社会的選択関数$f$を
  定義し,その性質を分析してきたが,その内容が適用されるのは何もゼミの形式決めという特定の
  文脈に限らない.ゼミ形式決定問題の定義を抽象化すれば
  \begin{itemize}
    \item $A$か$B$かという$2$択の多数決を
    \item $n(\geq 2)$人で
    \item $m(\geq 2)$回行う
  \end{itemize}
  ということになるので,ゼミの形式決めに限らず,同一の2択多数決を繰り返し行うような
  意思集約問題ならば同じ性質が成り立つ.

\section{まとめ}
本論文では,著者のゼミで実際に行われている実施形式の決定問題を題材にして,社会的選択理論の見地から
その性質を探求した.まず\ref{sec:問題設定の背景と社会的選択理論との関係}~節では社会的選択理論に
ついて紹介した.その中で,複数の個人の選好を集約して$1$つの結果を選び取るルールである社会的選択関数の
性質が重要になる事実を共有した.\ref{sec:多数決の問題点とベンチマークとしてのボルダルール}~節では
社会的選択関数の例として多数決とボルダルールを取り上げ,それらの利点と欠点を紹介した.
\ref{sec:戦略的操作の可能性}~節では耐戦略性に関する議論を紹介し,ボルダルールの欠点として前節で紹介された,
耐戦略性を満たさないという性質は,実はほぼすべての非独裁的な社会的選択関数が満たしてしまうことを述べた.一方で
個人のとりうる選好の集合を狭めて単峰性という性質が成り立つようなときには,この問題が回避できることも述べた.
\ref{sec:ゼミ形式決定問題の性質}~節では,それまでの議論を踏まえながら,ゼミの形式決定問題の性質を考察した.
まずはこれをゼミ形式決定手続きという名の社会的選択選択関数として定式化し,そのあとこの関数の性質に関する
いくつかの命題を証明した.特に重要だったのは,まずゼミ形式決定手続きがペア全敗者を選ばないこと.
次に,すべての個人の選好が回数に関する単峰性を満たすときには,中位回数だけ対面形式を行うような
選択肢が選ばれること.さらに同様の条件下で,回数に関する耐戦略性と言えるような,戦略的な操作を限定
する性質を持っているということである.

一方でできなかったことは,例えば,ゼミ形式決定手続きの定式化部分において,各$d_i$を決めるステップで$0$と$1$が
同数になったときにそれらをランダムで選ぶようにする場合である.この場合ゼミ形式決定手続きの値域は選択肢上の
確率分布となるので,その性質に関する議論も別に必要になってくる.

\newpage
\section*{謝辞}
現在執筆中...

\newpage
\begin{thebibliography}{9}
  \bibitem{招待}
  坂井豊貴 (2018)『社会的選択理論への招待 --- 投票と多数決の科学』日本評論社.
  \bibitem{メカデザ}
  坂井豊貴・藤中裕二・若山琢磨 (2020)『メカニズムデザイン --- 資源配分制度の設計とインセンティブ』ミネルヴァ書房.
  \bibitem{単峰性}
  Black, D. (1948) On the Rationale of Group Decision-making, \textit{Journal of Political Economy},
  Vol. 56, pp. 23--34.
  \bibitem{black}
  Black, D. (1976) Partial Justification of the Borda Count, \textit{Public Choice},
  Vol. 28, pp. 1--15.
  \bibitem{borda}
  Borda, J.-C. de (1784) M\'{e}moire sur les \'{e}lections au scrutin, \textit{Histoire de l'Acad\'{e}mie
  Royal des Sciences}, 1781, pp. 657--664.
  \bibitem{Coughlin}
  Coughlin, P. (1979) A Direct Characterization of Black's First Borda Count,
  \textit{Economic Letters}, Vol. 4, pp. 131--133.
  \bibitem{Farkas}
  Farkas, D. and Nitzan, S. (1979) The Borda Rule and Pareto Stability: A Comment,
  \textit{Econometrica}, Vol. 47, pp. 1305--1306.
  \bibitem{gibbard}
  Gibbard, A. (1973) Manipulation of voting schemes: A general result, \textit{Econometrica}, Vol. 41,
  pp. 587--601.
  \bibitem{malk}
    Malkevitch, J. (1990) Mathematical Theory of Elections, \textit{Annals of the New York
    Academy of Sciences}, pp. 89--97.
  \bibitem{revisit}
  Okamoto, N. and Sakai, T. (2019) The Borda rule and the pairwise-majority-loser
  revisited, \textit{Review of Economic Design}, Springer; Scioety for Economic Design, vol 23(1)
  , pp. 75--89, June.
  \bibitem{satterthwaite}
  Satterthwaite, M. A. (1975) Strategy-proofness and Arrow's Conditions:
  Existence and Correspondence Theorems for Voting Procedures and Social
  Welfare Functions, \textit{Journal of Economic Theory}, Vol. 10, pp. 187--217.
\end{thebibliography}


\end{document}