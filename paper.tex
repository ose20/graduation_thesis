\documentclass[dvipdfmx]{jsarticle}
%\setcounter{secnumdepth}{5}
\usepackage{otf}
\usepackage{array}
\usepackage{amsmath}
\usepackage{textcomp}
\usepackage{amssymb}
\usepackage{amsthm}

%ユーザ自身による体裁のカスタマイズが許される状況なら以下が使える.
%奥村氏のjsarticleクラスやjsbookクラスを用いた場合にはこれがデフォルトになってる.
%\DeclareFontShape{JY1}{mc}{m}{n}{<->jis}{}
%\DeclareFontShape{JY1}{gt}{m}{n}{<->jisg}{}
%\DeclareFontShape{JT1}{mc}{m}{n}{<->jis-v}{}
%\DeclareFontShape{JT1}{gt}{m}{n}{<->jisg-v}{}

\newtheorem{definition}{定義}
\newtheorem{proposition}[definition]{命題}
\newtheorem{example}{例}
\newtheorem*{THeorem}{定理}
\newtheorem*{PRoposition}{命題}
\newtheorem*{DEfinition}{定義}

\begin{document}

\title{ゼミ形式の決定手続きに見る,複数回に渡る同一2択投票問題が持つ社会的選択理論的性質}
\author{一橋大学経済学部 2117272C 横山彪人}
\date{\today}
\maketitle

\begin{abstract}
  著者の所属するゼミでは,その実施形式をオンラインにするか対面にするかを毎回メンバーによる
  投票によって決定している.これは複数人の選好を集約して1つの選択肢を選び取る社会的選択の問題である.
  
  本論文では,このゼミ形式の決定手続きが持つ意思集約方法としての性質を考察し,その結果を述べる.
  その足掛かりとして,はじめにボルダルールと戦略的操作に関する議論を紹介する.
  
  ボルダルールに関してはその利点と欠点について述べる.利点についてはペア全敗者を選ばないことを
  始めとした5つの利点を述べる.欠点については,結果に対する戦略的操作が可能なこと,
  さらに膨大な選択肢に対する選好を表明することが実用的でないことを述べる.
  
  次に,ボルダルールの欠点でも見られた戦略的操作に関する問題を,ボルダルール
  だけでなく意思集約方法一般に関する議論として紹介する.そこではまずギバート=サタスウェイト定理を
  引用してある条件を満たした非独裁的な意思集約方法は必ず戦略的操作が可能になってしまうことを述べる.次に
  選好の単峰性という概念を紹介し,すべての個人の選好がこの単峰性という条件を満たすときには戦略的操作を
  不可能にする意思集約方法が存在することを述べる.
  
  最後に,これらを踏まえた上でゼミの実施形式決定手続きが持つ意思集約方法としての性質を考察する.
  そのためにまずこの手続きを社会的選択関数として定式化する.この関数はまるで,各回の投票において各参加者が
  その時点で実現可能な選択肢のうち最も好ましいものに投票するかのようにふるまう.したがってこの関数は,
  最初に選択肢全体に関する選好を表明せずに各回ごとに投票を行うといった,実際に著者のゼミで行われている
  手続きの性質を反映している.そしてこの関数が次の3つの性質を満たすことを証明する.
  1つ目はペア全敗者を選ばないこと.2つ目は,2人以上の任意の人数と2回以上の任意のゼミ回数における手続きにおいて
  ボルダルールと異なる選択肢を選び取る選好組が存在すること.そして3つ目は,個々の選好が対面形式の実施回数に
  対して単峰性が成り立っているときは,その中位の回数に特徴付けられた選択肢が実現することである.
\end{abstract}

%目次の出力
\tableofcontents
\clearpage

\section{問題設定の背景と社会的選択理論との関係}
\subsection{問題設定の背景}
本論文が執筆された2020年は,COVID-19の感染拡大の影響で前期の講義がほぼすべてオンラインに移行した.しかし
後期になるとこの制限が緩和されたことで,ゼミに関しては対面形式での実施が可能になった.著者の所属するゼミでは
オンラインと対面を併用し,毎回次のゼミの実施形式のみをオンラインか対面かの2択投票の多数決で決めることになった.

この手続きは複数人の選好を集約して1つの選択肢を選び取る意思集約方法の1つである.そしてその方法の良し悪しを
分析するのは社会的選択理論の問題である.本論文執筆の動機は,著者が実際に使用している意思集約方法の性質を,
社会的選択理論的な見地から明らかにしたいという欲求による.

\subsection{社会的選択理論とは何か}\label{社会的選択とは何か}
坂井によれば,社会的選択理論というと「投票における意思集約方法の設計」と「社会状態の望ましさを評価する基準の構築」
を主に含むことが多い(坂井 2018, p. 3).そのうち本論文と関わりが深いのは前者の方である.つまり
2人以上の個人で1つの意思決定をする際に,どのように異なる選好同士の折り合いをつけ,意思を1つにまとめ上げるのが
よいのか,その意思集約方法の理論的性質に注目する.ここで,そのモチベーションを共有するために,意思集約方法の
性質が重要になる例を2つ述べる.

1つは多数決による例である.ある中学生のクラスで授業が休みになり,その時間を自由に使ってよい状況を考える.
ただし担任の教員には監督責任があるので,クラスの全員が同じ場所にいなければならない.このクラスには全員で30人
の生徒がおり,そのうち18人は体を動かして遊びたいと思っており,残る少数派の12人は教室で静かに読書をしたいと思っている.
このとき,「校庭でサッカーをして過ごす」という選択肢と「教室で読書をして過ごす」という選択肢で多数決を行えば,
前者が18人の指示を得て多数決に勝利する.一方でこの2つの選択肢に加えて「体育館でバスケットボールを
して過ごす」という選択肢が存在し,体を動かして遊びたい18人の生徒がサッカーとバスケットボールの選択肢に
半々に割れた場合には,多数決の勝者は「教室で読書をして過ごす」になる.この結果は全体における多数派である,
体を動かして遊びたい18人にとっては一番望ましくない選択肢であろう.これは似たような選択肢の存在によって
票の割れが発生し,結果として少数派の意見が採用される事例である.この場合,多数決は意思集約方法として
好ましいだろうか,そうでないならば,他に優れた方法はあるだろうか.

もう1つはMalkevitch(1990)によって提示された次の例で,意思集約方法の選択が最終的な結果を変えてしまうものである.
まず設定として,55人の有権者が5つの選択肢$a,b,c,d,e$について,表~\ref{tab:Malkevitch}のような
選好を持っているとし,意思集約方法として次の4つを考える.

1つ目は多数決である.この場合は$a$が18票を得て勝利する.

\begin{table}[h]
  \caption{Malkevitch(1990)の例.}\label{tab:Malkevitch}
  \begin{center}
    \begin{tabular}{c|c|c|c|c|c|c}
      & 18人 & 12人 & 10人 & 9人 & 4人 & 2人 \\ \hline
    1位 & a & b & c & d & e & e \\
    2位 & b & e & b & c & b & c \\
    3位 & e & d & e & e & d & d \\
    4位 & c & c & d & b & c & b \\
    5位 & b & a & a & a & a & a \\
\end{tabular}
  \end{center}
\end{table}

2つ目,通常の多数決をしたあと,その勝利者が過半数の指示を得ていない場合は2位と決選投票を行う方法を採用すればどうなるだろうか.
多数決で1位の$a$が獲得した18票は過半数ではないので,2位の$b$と決選投票が行われる.表~\ref{tab:Malkevitch}
によれば,$a$を$b$より好むものは18人おり,その逆は37人いるので,最終的には$b$が勝利する.これは多数決と
異なる選択肢を選び取っている.

3つ目は,毎回の多数決で最少票の選択肢を消去していく方法である.すると1段回目では6票しか集まらなかった
$e$が消去され,2段回目では9票しか集まらなかった$d$が消去され,3段回目では16票しか集まらなかった$b$
が消去され,最後の段階では18票しか集まらなかった$a$が消去されて$c$が残る.これは前の2つのいずれとも
異なる選択肢を選び取っている.

4つ目は,順位ごとに重みづけの得点を与えて,その総得点が一番高いものを選ぶ方法である.ここでは1位に5点,
2位に4点,\ldots,5位に1点を与えることにする.このとき,選択肢を引数にその総得点を返す関数を$p(\cdot)$
と表すと,それぞれの総得点は
\begin{align*}
  p(a) &= 5 \times 18 \ +\  4 \times 0 \ +\ 3 \times 0 \ +\  2 \times 0\ +\ 1 \times 37 = 127 \\
  p(b) &= 5 \times 12 \ +\  4 \times 14 \ +\ 3 \times 0 \ +\  2 \times 11 \ +\ 1 \times 18 = 156 \\
  p(c) &= 5 \times 10 \ +\  4 \times 11 \ +\ 3 \times 0 \ +\  2 \times 34 \ +\ 1 \times 0 = 162 \\
  p(d) &= 5 \times 9 \ +\  4 \times 18 \ +\ 3 \times 18 \ +\  2 \times 10 \ +\ 1 \times 0 = 191 \\
  p(e) &= 5 \times 6 \ +\  4 \times 12 \ +\ 3 \times 37 \ +\  2 \times 0 \ +\ 1 \times 0 = 189
\end{align*}
のように計算されるので,総得点が一番高いのは$d$となる.この結果は前の3つのいずれとも異なるものである.

この例が示すように,有権者の選好が変わらなくても,その集約方法を変えると最終的な結果も変わってきてしまう.
最初の中学生による多数決の例とこのMalkevitchの例が示唆するものは,どの集約方法を採用するべきか,
採用される集約方法はどのような性質を満たしているべきかという問題が重要だということである.

\subsection{本誌の構成}
2節では意思集約方法のベンチマークとして多数決とボルダルールを取り上げる.多数決に関してはペア全敗者を
選んでしまうという問題を抱えていることを述べる.ボルダルールに関しては,ペア全敗者を選ばないことをはじめ
とする利点といくつかの欠点について述べる.

3節では意思の集約結果に対する戦略的な操作の可能性について取り上げる.
まず有名な不可能性定理であるギバート=サタスウェイト定理を引用して,全射性という性質を持つ意思集約方法
に対して,戦略的操作を不可能にすることを求めるならばそれは独裁制しかないことを述べる.そのあと,
個々の選好が単峰性という条件を満たすときには,戦略的操作を不可能にするような非独裁的な集約ルールが
存在することを述べる.

4節では,2,3節の内容を踏まえながら,著者のゼミが採用している意思集約方法が持つ性質について
議論する.具体的にはこの方法を関数として定式化したあと,この方法がペア全敗者を選ばないこと,
メンバーが2人以上かつゼミの回数が2回以上ならボルダ勝者と異なる選択肢を選びうる選好組が存在すること,
そして各々の選好が回数に関する単峰性を持つときには,その回数の中位に特徴付けられた
選択肢が選ばれることを示す.

\section{多数決の問題点とベンチマークとしてのボルダルール}
\subsection{はじめに}
この2節では,意思集約方法の例示と4節でゼミ形式決定問題を定式化した際のベンチマークとして
多数決とボルダルールの2つを取り上げ,その性質について述べる.
\ref{subsec:多数決の問題点}では,多数決がペア全敗者を選んでしまうという問題点を,
ペア全敗者の定義と共に例を用いて解説する.続いて\ref{subsec:ボルダルール}ではボルダルールに
関する議論を紹介する.その中でまず\ref{subsubsec:ボルダルールの定義}ではボルダルールの定義を述べる.
\ref{ボルダルールの利点}ではボルダルールが持つ優れた性質を5つ紹介する.\ref{ボルダルールの欠点}
ではボルダルールの欠点を2つ述べる.

\subsection{多数決の問題点}\label{subsec:多数決の問題点}
多数決は,恐らく最も有名な意思集約方法の一つである.その方法は,各有権者が1つの選択肢に投票し,
最多票を得た選択肢が勝者になるというものである.しかしこの多数決がある問題を抱えていることは
Borda(1784)が指摘していた.彼が指摘したのは次のようなことである.今,3つの選択肢$x,y,z$に対する
21人の選好が表~\ref{ペア全敗者}のような状況を考える.多数決を行えば$x$が最多票の8票を得て勝者
となる.しかしここで1対1,つまりペアごとの多数決を行えば$x$は$y$に8対13で負け,$z$にも8対13で
負ける.ペアごとの多数決で全敗する$x$をペア全敗者\footnote{この命名は坂井(2018)に基づく}と呼ぶ.
多数決にはこのように,ペア全敗者を選びうるという性質がある.

\begin{table}[h]
  \caption{ペア全敗者を選ぶ多数決}\label{ペア全敗者}
  \begin{center}
    \begin{tabular}{c|c|c|c|c} \hline
      & 4人 & 4人 & 7人 & 6人 \\ \hline
      1位 & $x$ & $x$ & $y$ & $z$ \\
      2位 & $y$ & $z$ & $z$ & $y$ \\
      3位 & $z$ & $y$ & $x$ & $x$ \\ \hline 
    \end{tabular}
  \end{center}
\end{table}



\subsection{ボルダルール}\label{subsec:ボルダルール}
\subsubsection{ボルダルールの定義}\label{subsubsec:ボルダルールの定義}
多数決がペア全敗者を選びうるという問題を受けて,Bordaはボルダルールと呼ばれる次の集計方法を提案した.
それは,選択肢が$m$個あるときに,それぞれの有権者が1位に$m$点,2位に$m-1$点,\ldots,$m$位に1点を
与え,その総得点の一番高いものを選び取る方法である.\<\footnote{これは\ref{社会的選択とは何か}で述べた
Malkevitchの例の4つ目の方法である.}\ 
この総得点をボルダ得点,ボルダ得点の一番高い選択肢をボルダ勝者と呼ぶことにする.

選択肢を引数としてそのボルダ得点を返す関数を$p(\cdot)$と表すと,先の表の~\ref{ペア全敗者}における
$x,y,z$のボルダ得点はそれぞれ,
\begin{align*}
  p(x) = 3 \times 8 \ + \ 2 \times 0 \ + \ 1 \times 13 = 37 \\
  p(y) = 3 \times 7 \ + \ 2 \times 10 \ + \ 1 \times 4 = 45 \\
  p(z) = 3 \times 6 \ + \ 2 \times 11 \ + \ 1 \times 4 = 44
\end{align*}
となり,$y$が選ばれる.

\subsubsection{ボルダルールの利点}\label{ボルダルールの利点}
ボルダルールには様々な優れた性質がある.ここではそのうちの5つを簡単に紹介する.

まず1つ目に,ボルダルールはペア全敗者を選ばない.これは多数決がペア全敗者を選んでしまうことを受けて
提案された方法としては当然満たして欲しい性質である.証明は次の2つ目の性質の系として直ちに導かれる.

2つ目は,1対1の多数決でボルダ勝者が他のある選択肢に負ける場合,少なくとも1つの選択肢に1対1の多数決
で勝てるというものである(Okamoto and Sakai 2019).

3つ目はボルダルールに留まらずスコアリングルール全般に関わる定理である.ここでスコアリングルールとは,
より高い順位には高い得点を与えるように重みづけをして,その最終的な総得点の1番高いものを選び取るルールの
ことを指す.そしてこのスコアリングルールに関して,どんな有権者数でもペア全敗者を選ぶことがないスコアリングルール
はボルダルールのみであることが示されている(Okamoto and Sakai 2019).
つまりボルダルール以外のどんなスコアリングルールについても,ある人数とある選好組が存在して,
その元でペア全敗者を選んでしまうというわけである.
これは無数にあるスコアリングルールの中でボルダルールの唯一性を示す定理と言える.

4つ目は全員一致までの近さという観点から集約方法を分析し,全員一致まで最も近い選択肢がボルダルールに
よって選ばれるというものである(Farkas and Nitzan 1979).

そして5つ目は,1対1での多数決に注目し,その総当たり戦での平均得票率が最大になる選択肢をボルダルールが
選び取るというものである(Black 1976, Coughlin 1979).4つ目と同じでこちらも全員一致までの近さを測っているという見方ができる.
しかし4つ目は全体のランキング表における選択肢の位置を使用しているのに対して,この5つ目はあくまで
1対1の比較に基づいて全員一致までの近さを測っているところに違いがある.

\subsubsection{ボルダルールの欠点}\label{ボルダルールの欠点}
以上見たようにボルダルールには様々な優れた性質があるが,問題点もある.ここではそのうち2つを述べる.

まず,ボルダルールは参加者に対してすべての選択肢に対するランクづけを要請する.選択肢が3つや
4つのときには問題ないかもしれないが,選択肢の数がさらに増えたときにはこのランクづけは参加者にとって
ひどく億劫な作業になる.特に本論文で考えたいゼミ形式の決定に関する問題では,ゼミの回数が$m$回として
各回において対面かオンラインかの2択があるから,
選択肢の数は$2^m$個にものぼる.後期のゼミは少なくとも10回はあるので選択肢の数は1000を超えることになる.
このような膨大な量の選択肢に対する選好を表明させるのは現実的ではない.したがってボルダルールをそのまま
ゼミ形式の決定問題に適用することはできない.

次に,ボルダルールは戦略的操作の影響を受ける.これは坂井(2018)の例を用いて次のように
説明される.今,4つの選択肢$x,y,z,w$に対して,3人の個人が表~\ref{Sakai2018}のような
選好を持っているとする.このときの4つのボルダ得点はそれぞれ,$p(x) = 10,\ p(y) = 11,
\ p(z) = 6, \ p(w) = 3$となりボルダ勝者は$y$である.ところがここで,$y$よりも$x$を
高く順序づけている佐藤が1位から順に$x,z,w,y$という虚偽の選好を表明すれば,4つのボルダ得点は
それぞれ,$p(x) = 11,\ p(y) = 9,\ p(z) = 7, \ p(w) = 4$となり,佐藤にとってより望ましい
$x$を実現することができる.

\begin{table}[h]
  \caption{坂井(2018)の例}\label{Sakai2018}
  \begin{center}
    \begin{tabular}{c|c|c|c} \hline
        & 佐藤 & 高橋 & 中野 \\ \hline
      $1$位 & $x$ & $y$ & $y$ \\
      $2$位 & $y$ & $x$ & $x$ \\
      $3$位 & $z$ & $z$ & $z$ \\
      $4$位 & $w$ & $w$ & $w$ \\ \hline
    \end{tabular}
  \end{center}
\end{table}

今見たように,ボルダルールには,虚偽の選好を表明することで自分にとってより好ましい結果を
選ばせることができるような状況が存在する.逆に,いかなる状況,いかなる個人においても,虚偽の選好を
表明することで結果をより好ましいものに変えることができないとき,その意思集約方法は
耐戦略性を満たすという.ボルダルールは耐戦略性を満たさないということだ.


\section{戦略的操作の可能性}
\subsection{はじめに}
\ref{ボルダルールの欠点}ではボルダルールの結果に対する戦略的操作が可能な場合があること,つまり
ボルダルールが耐戦略性を満たさないという問題があることを述べた.この3節では,耐戦略性に関する議論を
ボルダルールだけでなく一般の意思集約方法にまで対象を広げる.\ref{subsec:ギバート=サタスウェイト定理}では,
ギバート=サタスウェイト定理を引用して,全射性という弱い条件を満たす意思集約方法に対戦略性を求めるならば
その方法は独裁制しかないことを述べる.\ref{subsec:単峰性と中位選択関数}では,すべての個人の選好が
単峰性という条件を満たすならば,耐戦略性を満たす非独裁的な意思集約方法が存在することを述べる.この単峰性に
関する発想は4節で応用される.

\subsection{ギバート=サタスウェイト定理}\label{subsec:ギバート=サタスウェイト定理}
\ref{ボルダルールの欠点}で述べたように,ボルダルールは耐戦略性を満たさない.
しかしこの問題点はボルダルールだけでなく,ほぼすべての民主的な意思集約方法に当てはまることが
Gibbard(1973)やSatterthwaite(1975)によって知られている.ここではギバート=サタスウェイト定理
として知られるその成果について述べる.尚,以降定義などで用いる記法は坂井他(2020)に基づいた.

\begin{DEfinition}
  $I$を個人の集合,$X$を選択肢の集合,$X$上の選好の集合を$\mathcal{R}$,
  $X$上の強選好の集合を$\mathcal{P}$とする.各$i \in I$の選好を$\succsim_i$で表す.
  個人$i$が取りうる選好の集合を$\mathcal{D}_i \subset \mathcal{R}$と表し,これを
  選好集合と呼ぶ.個人の選好集合の直積
  \[
    \mathcal{D}_I \stackrel{\mathrm{def}}{=} \mathcal{D}_1 \times
    \mathcal{D}_2 \times \cdots \times \mathcal{D}_n
  \]
  をドメインと呼び,個人の選好組を
  \[
    \succsim \stackrel{\mathrm{def}}{=} (\succsim_1, \succsim_2, \ldots, \succsim_n)
      \in \mathcal{D}_I
  \]
  で表す.このとき社会的選択関数(今まで集約方法と呼んでいたもの)は次のように表せる.
  \[
    {\mathnormal{f}}\colon \mathcal{D}_{\mathrm{I}} \to \mathrm{X}
  \]
\end{DEfinition}

今定義された言葉を用いて,社会的選択関数の様々な性質もまた定義することができる.まず,
先に述べた集約方法の耐戦略性は次のように書ける.

\begin{DEfinition}[耐戦略性]
  社会的選択関数$f$が耐戦略性を満たすとは,すべての$i \in I$,$\succsim\ \in \mathcal{D}_I$,
  $\succsim_{i}'\ \in \mathcal{D}_i$について
  \[
    f(\succsim) \succsim_{i} f(\succsim_{i}',\succsim_{-i})
  \]
  が成り立つことである.
\end{DEfinition}

それから独裁制と全射性についても定義を与える.

\begin{DEfinition}[独裁制]
  社会的選択関数$f$が独裁制であるとは,ある個人$i \in I$が
  存在し,すべての$\succsim\  \in \mathcal{D}_I$について
  \[
    f(\succsim)\succsim_i x\hspace{8pt} \forall x \in \mathrm{X}  
  \]
  が成り立つことである.この$i$を独裁者という
\end{DEfinition}

\begin{DEfinition}[全射性]
  社会的選択関数が全射性を満たすとは,任意の$x \in \mathrm{X}$について,ある$\succsim \ \in \mathnormal{D}_{\mathrm{I}}$
  が存在して$\mathnormal{f}(\succsim) = x$
  が成り立つことである
\end{DEfinition}

以上の定義によって,ギバート=サタスウェイト定理を述べることができる.

\begin{THeorem}[ギバート=サタスウェイト定理]
  $X = A$,$|A| \geq 3$とする.社会的選択関数
  $f\colon \mathcal{P}^{I} \to X$が全射性を満たすとき以下が成り立つ.
  \begin{center}
    $f$は耐戦略性を満たす $\Leftrightarrow$ $f$は独裁制である.
  \end{center}
\end{THeorem}


\subsection{単峰性と中位選択関数}\label{subsec:単峰性と中位選択関数}
本論文における耐戦略性の議論は,ボルダルールがこれを満たさないということから始まった.しかし,
一般的に言って,社会的選択関数に耐戦略性を要請することは極めて難しいということが
ギバート=サタスウェイト定理で示される.

ただし,個人の選好が単峰性という特別な条件を満たすときには,耐戦略性を満たす非独裁的な
社会的選択関数が存在することがBlack(1948)で指摘されている.ここでは単峰性とその関数に関する
議論を紹介し,次節でゼミの形式決定問題に援用する.

\begin{DEfinition}[単峰性]
  選択肢の集合を$A = \{a_1, a_2, \ldots, a_{l}\}$と表し,各選択肢は区間$[0,1]$内の数値として
  与えられ,
  \[
    0 \leq a_1 < a_2 < \cdots < a_{l} \leq 1
  \]
  を満たすとする.$A$上の選好$\succsim_i \ \in \mathcal{R}$が単峰的であるとは,ベストの選択肢
  $b(\succsim_i) \in A$がただ一つ存在して,任意の$k, k' \in \{ 1, 2, \ldots, \mathnormal{l} \}$
  について
  \[
    [a_k < a_{k'} < b(\succsim_i) \text{または} b(\succsim_i) < a_{k'} < a_k]
    \Rightarrow b(\succsim_i) \succ_i a_{k'} \succ_i a_{k}
  \]
  を満たすことである.また,$A$上の単峰的な選好すべてからなる集合を$\mathcal{T}$と書く
\end{DEfinition}

この単峰性の直観的な理解は次のようなものである.つまり選択肢をなんらかの基準で直線状に並べることができ,
その中で最も好ましいものが存在する.各選択肢は,この最も好ましいものから離れれば離れるほど
望ましくなくなっていくというものだ.単峰性が成り立ちそうな選択肢の例としては,税率,売り手にとっての商品の値段,
部屋の室温などが考えられる.

すべての個人の選好が単峰的であるとき,中位選択関数を定めることができる.そしてこの中位選択関数が
耐戦略性を満たす社会的選択関数である.

\begin{DEfinition}[中位選択関数]
  任意の$i$に対して$\mathcal{D}_i \subset \mathcal{T}$であるような
  選好組$\succsim \ \in \mathcal{D}_I$に対して,$b^m(\succsim)$を
  $b(\succsim_1)$,$b(\succsim_2)$,\ldots,$b(\succsim_n)$における中位の選択肢として定める.
  つまり,$b^m(\succsim) \in \{ b(\succsim_i):i \in I \}$かつ
  \begin{eqnarray*}
    | \{i \in I: b(\succsim_i) \leq b^m(\succsim) \} | \geq \frac{n}{2} \\
    | \{i \in I: b(\succsim_i) \geq b^m(\succsim) \} | \geq \frac{n}{2}
  \end{eqnarray*}
  が成り立つ.$n$が偶数の場合は中位が$2$つ存在するケースがあるが,その時は常に左側(または右側)の中位
  を選ぶことにする.このとき中位選択関数$f^m\colon \mathcal{D}_{I} \to X$
  を,$f^m(\succsim) \stackrel{\mathrm{def}}{=} b^m(\succsim)$と定める.
\end{DEfinition}

\begin{THeorem}
  任意の$i$に対して$\mathcal{D}_i \subset \mathcal{T}$であるとする.このとき中位選択関数
  $f^m\colon \mathcal{D}_I \to A$は耐戦略性を満たす.
\end{THeorem}

\section{ゼミ形式決定問題の性質}
\subsection{ゼミ形式決定問題の定式化}
この節では,著者のゼミで行っている実施形式の決定問題の性質について述べる.実際に行われている手続きは,
$k$を自然数として,$k-1$回目のゼミ終了時に$k$回目のゼミの実施形式をオンラインか対面かで2択の投票
を行うというものである.これ以外のルールはないので,議論を始める前に正式な定式化を与える必要がある.

ここからは,実際のゼミ形式決定問題をある社会的選択関数として定式化し,その上で議論を進めていく形をとる.
また,これから定義するその社会的選択関数を,便宜上「ゼミ形式決定手続き」と呼ぶことにする.

はじめに,定式化のために必要ないくつかの検討と,記法の導入を行う.

まず,ゼミの人数を$n$で表し,さらに$n \geq 2$であるとする.この仮定を置く理由は,分析したい問題が複数人の
意思決定に関するものだからである.また,ゼミの実施回数を$m$で表し,$m \geq 2$であるとする.これは,$m=1$
だとただの2択の多数決になって面白くないからである.

次に,選択肢の集合を定義する.個人が取りうる選択肢は,各回においてオンラインか対面かの2択に分かれるので
$2^m$個存在する.この性質を扱いやすいように選択肢の集合$A$を以下のように定義する.

\begin{definition}[選択肢の集合]
  ゼミ形式決定手続きの選択肢の集合$A$を以下で定義する.
  \[
    A \stackrel{\mathrm{def}}{=} \{00\cdots 00,\ 00\cdots 01,\ 00\cdots 10,\ 00\cdots 11,\
    \ldots,\ 11\cdots 11\}
  \]
  ただし,$A$の要素は$0 \leq a \leq 2^m-1$を満たす$a$を$2$進数で表したものである.さらに$a \in A$
  について,$a$の左から$k$桁目$(k \in \{1,2,\ldots,m \})$の数字が$0$なら$k$回目のゼミをオンライン
  で行い,同様に左から$k$桁目の数字が$1$なら$k$回目のゼミを対面で行うような選択肢を表す.
\end{definition}

\begin{example}
  $m=3$のとき,$A = \{000,\ 001,\ 010,\ 011,\ 100,\ 101,\ 110,\ 111 \}$である.このうち,$101 \in A$は
  $1$回目と$3$回目のゼミを対面形式で実施し,$2$回目のゼミをオンライン形式で実施するような選択肢である.
\end{example}

ゼミ形式決定手続きでは,全$m$回のうち特定の回の実施形式だけに注目したい場合がある.そのために以下の記法を導入する.

\begin{definition}[桁の抽出]
  $a \in A$,$k \in \{1,2,\ldots,m \}$について
  \[
    a(k) \stackrel{\mathrm{def}}{=} a\text{の左から}k\text{桁目の数}
  \]
  とする.
\end{definition}

また,1回ごとに実施形式を決めていく手続きの性質上,ある回までの結果次第ではその時点で
実現不可能な選択肢が存在する.例えば,$m=3$で1回目の実施形式がオンラインになった場合,
1回目に対面形式を行うような選択肢,つまり$A$の部分集合
\[
  \{100, 101, 110, 111 \}
\]
の要素はもはや実現不可能である.このようにある時点での実現可能・不可能な選択肢の集合を表すために
以下の記法を導入する.

\begin{definition}
  任意の$k \in \{1,2,\ldots,m-1\}$と,任意の$d_{1}d_{2}\cdots d_{k}$(ただし$j \in \{1,2,\ldots,k\}$に対して
  $d_j$は$0$または$1$をとる)に対して,選択肢の集合$A$の部分集合$A^{d_{1}d_{2}\cdots d_{k}}$を以下で定める.
  \[
    A^{d_{1}d_{2}\cdots d_{k}} \stackrel{\mathrm{def}}{=} \{a \in A |\ 
      \forall i \in \{1,\ldots,k\} \ a(i) = d_i \}
  \]
  とする.
\end{definition}


\begin{example}
  $m=5$とする.このとき,$A^{101}$は$\{10100,\ 10101,\ 10110,\ 10111\}$である.
\end{example}

個人$i$の選好$\succsim_i$についてはこの節では$A$上の強選好$\mathcal{P}$を取ると仮定する.
すると,$A$の任意の部分集合に対して,その中で最も好ましい選択肢がただ一つ存在する.
この関係について,特に以下のような定義を与える.

\begin{definition}
  任意の$i \in I$,任意の$k \in \{1,2,\ldots,m-1\}$,さらに任意の$d_{1}d_{2}\cdots d_{k}$
  (ただし$j \in \{1,2,\ldots,k \}$に対して$d_j$は$0$または$1$をとる)に対して$b_{i}$と
  $b_{i}^{d_{1}d_{2}\cdots d_{k}}$を以下のように定める.

  \begin{align*}
    b_i &\stackrel{\mathrm{def}}{=} b \in A\ \text{かつ}\ \forall a \in A. \ b \succsim_i a\
    \text{を満たす唯一の}b \\
    b_{i}^{d_{1}d_{2}\cdots d_{k}} &\stackrel{\mathrm{def}}{=}
    b \in A^{d_{1}d_{2}\cdots d_{k}} \ \text{かつ}\ \forall a \in A^{d_{1}d_{2}\cdots d_{k}}.
    \ b \succsim_i a\ \text{を満たす唯一の}b
  \end{align*}
\end{definition}

この定義の$b_i$は,選択肢集合$A$の中で個人$i$にとって最も好ましい選択肢を表している.
同様に$b_{i}^{d_{1}d_{2}\cdots d_{k}}$は,$A$の部分集合である$A^{d_{1}d_{2}\cdots d_{k}}$の中で
個人$i$にとって最も好ましい選択肢を表している.

\begin{example}
  $m=3$で,個人$i$の選択肢集合$A$に対する選好$\succsim_i$が
  \[
    101 \succsim_i 011 \succsim_i 110 \succsim_i 010 \succsim_i 001 \succsim_i 100
    \succsim_i 111 \succsim_i 000
  \]
  であるとき,$b_i = 101$,$b_{i}^{0} = 011$,$b_{i}^{00} = 001$である.
\end{example}

以上の定義を用いて,ゼミ形式決定手続きを社会的選択関数として定義することができる.
\begin{definition}[ゼミ形式決定手続き]
  ゼミ形式決定手続きとは,以下の条件を満たす関数$f\colon \mathcal{P}^I \to A$である.\\
  任意の$\succsim \ \in \mathcal{P}^I$に対して$f(\succsim) = d_{1}d_{2}\cdots d_{m} \in A$と定める.
  ただし$k \in \{2,3,\ldots,m\}$として
  \begin{align*}
      d_{1}
    &= \left\{ \begin{array}{@{\,}rl@{}}
      1 & \mbox{if}\hspace{5mm}  |\{i \in I\ |\ b_i(1) = 1\}| \hspace{2mm} \geq \hspace{2mm} |\{i \in I\ |\ b_i(1) = 0\}| \\
      0 & otherwise
    \end{array} \right. \\
      d_{k}
    &= \left\{ \begin{array}{@{\,}rl@{}}
      1 & \mbox{if}\hspace{5mm}  |\{i \in I\ |\ b_i^{d_{1}d_{2}\cdots d_{k-1}}(k) = 1\}| \hspace{2mm}
      \geq \hspace{2mm} |\{i \in I\ |\ b_i^{d_{1}d_{2}\cdots d_{k-1}}(k) = 0\}| \\
      0 & otherwise
    \end{array} \right.
  \end{align*}
\end{definition}

このゼミ形式決定手続きの直観的理解は次のようなものである.まず各回の実施形式は,1回目から順番に
オンラインか対面かの2択問題によって決められていく.各回ではその時点で実現可能な選択肢のうち,
個人$i$にとって最も好ましい選択肢の情報が使われる.例えば1回目はすべての選択肢が実現可能なので,
この時点で個人$i$にとって最も好ましい選択肢は$b_i$である.この手続きは,1回目の投票であたかも個人$i$が
$b_i$を成し遂げるために$b_{i}(1)$に投票したかのように対面かオンラインかの多数決を行う.
尚,両者の票が同数だった場合は対面\footnote{オンラインにしても以降の議論には多少の記号の変化を除いてほとんど影響はない.
ただし,必ずどちらか一方を選ぶように決めておく.}にする.$k(\geq 2)$回目では,その時点で実現可能な選択肢の集合は
$A^{d_{1}d_{2}\cdots d_{k-1}}$であるから,ゼミ形式決定手続きは各個人$i$にとってこの集合の中で最も
好ましい選択肢$b_i^{d_{1}d_{2}\cdots d_{k-1}}$の情報を用いて多数決を行う.つまりこの手続きは,
$k$回目の投票であたかも個人$i$が$b_i^{d_{1}d_{2}\cdots d_{k-1}}$を成し遂げるために
$b_i^{d_{1}d_{2}\cdots d_{k-1}}(k)$に投票したかのようにこの情報を用いて対面かオンラインかの多数決を行う.

\ref{ボルダルールの欠点}で述べたように,個人に$2^m$個の選択肢に対する選好を表明させることは現実的ではない.
それは今定義したゼミ形式決定手続きに対しても同じことが言える.事実,実際に著者のゼミで行われているゼミの形式決定でも,
一度にすべての回の形式を決めることはせずに,各回ごとにそれぞれが次回の実施形式を2択で投票して多数決で決めているのだった.
しかしここですべての個人が,各回の投票において,その時点で実現可能な選択肢のうちで最も好ましい選択肢が
実現されるように投票を行っているとしたら,その最終的な帰結はゼミ形式決定手続きと全く同じものになる.








\subsection{ボルダルールとの比較}
\subsection{単峰性との関連性}

\section{まとめ}

\section*{謝辞}
ありがとうございました.


\begin{thebibliography}{9}
  \bibitem{招待}
  坂井豊貴 (2018)『社会的選択理論への招待』日本評論社.
  \bibitem{メカデザ}
  坂井豊貴他 (2020)『メカニズムデザイン』ミネルヴァ書房.
  \bibitem{単峰性}
  Black, D. (1948) On the Rationale of Group Decision-making, Journal of Political Economy,
  Vol. 56, pp.23-34.
  \bibitem{black}
  Black, D. (1976) Partial Justification of the Borda Count, Public Choice,
  Vol. 28, pp. 1-15.
  \bibitem{Coughlin}
  Coughlin, P. (1979) A Direct Characterization of Black's First Borda Count,
  Economic Letters, Vol. 4, pp. 131-133.
  \bibitem{borda}
  de Borda J-C (1784) M\'{e}moire sur les \'{e}lections au scrutin, Histoire de l'Acad\'{e}mie
  Royal des Science, pp. 657-664.
  \bibitem{Farkas}
  Farkas, D. and Nitzan, S. (1979) The Borda Rule and Pareto Stability: A Comment,
  Econometrica, Vol. 47, pp. 1305-1306.
  \bibitem{gibbard}
  Gibbard, A. (1973) Manipulation of voting schemes: A general result, Econometrica, Vol. 41,
  pp. 587-601.
  \bibitem{malk}
    Malkevitch, J. (1990) Mathematical Theory of Elections, Annals of the New York
    Academy of Science, pp. 89-97.
  \bibitem{revisit}
  Okamoto, N. and Sakai, T. (2019) The Borda rule and the pairwise-majority-loser
  revisited, Review of Economic Design, Springer; Scioety for Economic Design, vol 23(1)
  , pages 75-89, June.
  \bibitem{satterthwaite}
  Satterthwaite, M.A. (1975) Strategy-proofness and Arrow's conditions:
  Existence and correspondence theorems for voting procedures and social
  welfare functions, Journal of Economic Theory, Vol. 10, pp. 187-217.
\end{thebibliography}


\end{document}